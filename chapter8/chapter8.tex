\addtocontents{toc}{\protect\newpage}
\chapter{Conclusion and Further Work}
\label{chap:Conclusion}
	%\note
	%{
	%	Well that was a load of nonsense wasn't it? Let's go to the pub!
	%}

	In this	thesis we have investigated the problem of applying timbral control using harmonic excitation systems. The
	primary aim was to identify techniques by which harmonic excitation can be used to provide intuitive control over
	the perceived properties of audio signals. These can then be used to aid novice users in carrying out common audio
	engineering tasks. Harmonic excitation techniques are useful in these applications as they allow for a greater
	number of spectral features to be manipulated than LTI systems; they can be used to introduce new frequency
	content to a signal. A disadvantage is that they are typically nonlinear system and their effects are highly
	dependant on the content of the input signal. A key problem to solve for this work was the development of harmonic
	excitation systems which produce similar effects over a wide range of input signals.

	The approach taken in this thesis can be divided into three main stages. Firstly, subjective data about the
	manipulation of timbre in music production was gathered. Various analysis techniques were then applied to this data
	to determine how language is used to describe timbral transformations and the low level audio features which
	correlate with these descriptions. Secondly, a number of nonlinear and time variant signal processing algorithms
	were evaluated against how well their effects on the features of a signal can be controlled. The results from this
	analysis were used to inform the design of harmonic excitation systems which provide monotonic control over a
	particular low level audio feature. Lastly, these systems were specialised to complete common audio processing
	tasks: reconstructing the spectrum of degraded audio signals and applying creative timbral manipulations using
	semantically labelled parameters. The performance of the resulting systems was evaluated with a series of perceptual
	listening tests. Sections \ref{sec:Conclusion-Descriptors} to \ref{sec:Conclusion-TimbralControl} summarise the main
	findings from each of these stages.

\section{Timbral Descriptors}
\label{sec:Conclusion-Descriptors}
	\note
	{
		\begin{itemize}
			\item Timbral groups for equalisation and distortion.
			\item Some terms are specific to an effect while others are used across both.
			\item Differences between terms defined by absolute feature values and those which describe a
				transform.
			\item Most salient features and correlations.
		\end{itemize}
	}

\section{Control of Audio Features with Harmonic Exciters}
\label{sec:Conclusion-FeatureConrol}
	\note
	{
		\begin{itemize}
			\item Nonlinear systems must be configured such that their effects are predictable for a large
				number of inputs.
			\item Individual control provides the most flexibility. (IAP, SSBA).
			\item Some features can be controlled more efficiently by simpler systems.
			\item Isolation of $f_{0}$ guarantees harmonic partials in output for most systems. Others the
				$f_{0}$ must also be tracked well.
			\item Control is provided monotonically but not linearly.
		\end{itemize}
	}

\section{Timbral Control with Harmonic Exciters}
\label{sec:Conclusion-TimbralControl}
	\note
	{
		Make this the longest!

		\begin{itemize}
			\item IAP performs well at generating individual harmonics and is also positive homogeneous.
			\item Bright and harsh are easy to replicate using the designed systems, crunch and warmth are more
				difficult. Perhaps due to poor definitions and use of full wave rectification.
			\item Put in percentages and whatnot.
		\end{itemize}
	}

\section{Critique}
\label{sec:Conclusion-Critique}
	\note
	{
		Critical review? (what do we cover well and what do we miss out, pros and cons)

		Say why it's good before slagging it off.

		Set up things for further work.

		\begin{itemize}
			\item Only works on tonal features.
			\item Limited sample size for all descriptors other than warm and bright.
			\item Vocabulary elicited from trained music producers who may have learned them, thereby not being
				useful for untrained people.
			\item Control of features is not independent. (monotonic but not independent)
			\item No information on extent to which a term applies to a transform / signal. The differences
				found between groups could be much smaller than in reality. No ability to rank signals /
				transforms in terms of a descriptor.
		\end{itemize}
	}

\section{Further Work}
\label{sec:Conclusion-FurtherWork}
	\note
	{
		\begin{itemize}
			\item Extend to other descriptors.
			\item Run more extensive experiments to highlight the differences between the terms within the
				timbral groups.
			\item Potential applications in audio restoration.
			\item Alter SAFE to elicit the degree to which a term applies.
			\item Apply to other classes of effects.
		\end{itemize}
	}
