\addtocontents{toc}{\protect\newpage}
\chapter{Conclusion and Further Work}
\label{chap:Conclusion}
	\note
	{
		Well that was a load of nonsense wasn't it? Let's go to the pub!
	}

\section{Control of Audio Features with Harmonic Exciters}
\label{sec:Conclusion-FeatureConrol}
	\note
	{
		\begin{itemize}
			\item Nonlinear systems must be configured such that their effects are predictable for a large
				number of inputs.
			\item Individual control provides the most flexibility. (IAP, SSBA).
			\item Some features can be controlled more efficiently by simpler systems.
			\item Isolation of $f_{0}$ guarantees harmonic partials in output for most systems. Others the
				$f_{0}$ must also be tracked well.
			\item Control is provided monotonically but not linearly.
		\end{itemize}
	}

\section{Timbral Descriptors}
\label{sec:Conclusion-Descriptors}
	\note
	{
		\begin{itemize}
			\item Timbral groups for equalisation and distortion.
			\item Some terms are specific to an effect while others are used across both.
			\item Differences between terms defined by absolute feature values and those which describe a
				transform.
			\item Most salient features and correlations.
		\end{itemize}
	}

\section{Timbral Control with Harmonic Exciters}
\label{sec:Conclusion-TimbralControl}
	\note
	{
		\begin{itemize}
			\item IAP performs well at generating individual harmonics and is also positive homogeneous.
			\item Bright and harsh are easy to replicate using the designed systems, crunch and warmth are more
				difficult. Perhaps due to poor definitions and use of full wave rectification.
			\item Put in percentages and whatnot.
		\end{itemize}
	}

\section{Critique}
\label{sec:Conclusion-Critique}
	\note
	{
		\begin{itemize}
			\item Limited sample size for all descriptors other than warm and bright.
			\item Vocabulary elicited from trained music producers who may have learned them, thereby not being
				useful for untrained people.
			\item Control of features is not independent.
			\item No information on extent to which a term applies to a transform / signal. The differences
				found between groups could be much smaller than in reality. No ability to rank signals /
				transforms in terms of a descriptor.
		\end{itemize}
	}

\section{Further Work}
\label{sec:Conclusion-FurtherWork}
	\note
	{
		\begin{itemize}
			\item Extend to other descriptors.
			\item Run more extensive experiments to highlight the differences between the terms within the
				timbral groups.
			\item Potential applications in audio restoration.
			\item Alter SAFE to elicit the degree to which a term applies.
			\item Apply to other classes of effects.
		\end{itemize}
	}
