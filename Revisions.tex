\documentclass[a4paper]{article}
\usepackage[cm]{fullpage}
\usepackage{array}
\usepackage{longtable}

\author{Sean Enderby}
\title{Thesis Revisions \\
	\Large Timbral Control of Audio Through Harmonic Excitation}
\date{}

\renewcommand{\familydefault}{\sfdefault}

\newcolumntype{C}[1]{>{\centering}p{#1}}
\newcommand{\answer}[1]{\underline{\underline{#1}}}

\begin{document}
\maketitle

\begin{center}
	\begin{longtable}{|c|C{8cm}|C{8cm}|}
		\hline 
		\bf{No.} & \bf{Revision} & \bf{Action Taken} \tabularnewline
		\hline
		1. & Review the abstract to ensure that the contribution to knowledge is more explicit & \tabularnewline
		\hline
		2. & Review the acronym list for completeness & Entries for ADSR, AMDF, API, AU, EQ, ERB, HPS, LFO, MUSHRA,
			RDF, SAFE and VST added to the acronym list \tabularnewline
		\hline
		3. & The mathematical notation list lacks Mahalanobis distance & \tabularnewline
		\hline
		4. & In Chapter 1 add flowchart of key stages of audio development process & \tabularnewline
		\hline
		5. & Add PCA equation on p26 & \tabularnewline
		\hline
		6. & Fig 5.12 should be moved to the appropriate place in Chapter 3 & Figure moved to Section~3.5.8 and
			accompanying text in that section altered. Relevant text in the spectral folding subsection of
			Section~5.2.4 also altered. \tabularnewline
		\hline
		7. & Amend axis labelling on Figs 3.4. 3.5 and the former Fig 5.12 to add correct frequency figures &
			Figures~3.4 to 3.6 now have frequency axis labels at 0 and $\frac{f_{s}}{2}$ \tabularnewline
		\hline
		8. & In s4.2.2, adjust motivation for plugins & Extended the penultimate paragraph of
			Section~4.2.2 to mention the additional APIs provided by some DAWs.\tabularnewline
		\hline
		9. & In s4.2.3, add justification for 5-second plugin sample size & Altered the second
			paragraph of Section~4.2.3 to mention engineers configuring effect using short loops of audio
			as discussed in the Viva.\tabularnewline
		\hline
		10. & On pp68-70, explain how much variance is explained by principal components & First
			paragraph of each of the subsections in Section~4.3.5 altered to include the proportion of
			total variance described by the timbre space. \tabularnewline
		\hline
		11. & On the plots with principal components in Chapter 4, ensure that numbering on axes is complete &
			\tabularnewline
		\hline
		12. & In chapter 4, explain what happens to PC4 and PC5 & Note about the lack of patterns in PCs 4 and 5
			included in the same paragraphs as changed for revision 10. \tabularnewline
		\hline
		13. & On p96 equation 5.13, the derivation for integrator for series should be included (eg in an Appendix
			as discussed below) & Derivation of Fourier series coefficients given in Appendix~A.2.  Derivation
			of the Fourier series for full wave rectification (Equation~5.11) also included as Appendix~A.1 for
			completeness.  References added to the text in the relevant sections, linking to the appendices.
			\tabularnewline
		\hline
		14. & Figs 5.8-5.11 and 5.14 need replotting for greater clarity & Unprocessed signal moved to its
			own figure (Figure~5.8), the processed signals for the multiplier, SSBA and IAP techniques then
			shown on their own in Figures~5.9 to 5.12. Spectra of signal processed by STTR split into
			subfigures in Figure~5.14 \tabularnewline
		\hline
		15. & On p99, the description of IAP should mention and reference its relationship to complex FM synthesis &
			First paragraph of the IAP subsection of Section~5.2.4 altered to reference Le Brun describing the
			spectrum of complex FM.  \tabularnewline
		\hline
		16. & s5.2.5 should mention the square wave appearance of the signal & Second paragraph of Section~5.2.5
		altered to mention square waves. \tabularnewline
		\hline
		17. & On p100, correct the cutoff frequency reference at top of page & Expression corrected to
			$\left( \frac{f_{2}}{2} - f \right)$Hz. \tabularnewline
		\hline
		18. & In Chapter 6, ensure complete descriptions of the filters used are given (eg in an Appendix as
			discussed below) & \tabularnewline
		\hline
		19. & On p118, place a methodology discussion within the dedicated evaluation methodology section and
		describe and justify test signals used in the final paragraph & Methodology section added (Section~6.3.1),
		more detail added about the parameterisation process (mentioning derivation of parameter / feature
		relationship formulae). \tabularnewline
		\hline
		20. & On p125, give full derivation of equation 6.6 (eg in an Appendix as discussed below) &
			Derivation of spectral flatness parameterisation given in Appendix~A.3. Reference to that appendix
			added to the text. \tabularnewline
		\hline
		21. & Be consistent in use of tristimulus/tristimuli & \tabularnewline
		\hline
		22. & Table 6.1 (p135) should be moved to s6.3 & Table moved to its own section (Section~6.3.1).
			Second paragraph in the conclusion section of Chapter~6 adjusted accordingly. \tabularnewline
		\hline
		23. & The first 2 sentences of Chapter 7 can be deleted & Deleted those two sentences. \tabularnewline
		\hline
		24. & Expand the second paragraph of Chapter 7 to explain the experimental methodology used &
			\tabularnewline
		\hline
		25. & In s7.2.2, insert a flow diagram to clearly explain operation of synthesis systems & \tabularnewline
		\hline
		26. & Explain the colour coding, and increase the size of, Figs 7.4 and 7.6 & Figure size increased
			and the text of the first paragraph in the Subjective Evaluation subsection of Section~7.2.2
			amended to explain the use of colour. \tabularnewline
		\hline
		27. & In s7.3.4, clarify and explain the derivation of percentages and their relationship with heat maps &
			\tabularnewline
		\hline
		28. & In s8.4.2, improve the clarity of the second paragraph to deal separately with monotonic and
			polyphonic signals, perhaps as 2 paragraphs & \tabularnewline
		\hline
		29. & The section on further work needs to identify future work directly emerging from the thesis &
			\tabularnewline
		\hline
		30. & Throughout, review the use of `conclusion', `conclusions' and `summary', particularly at the end of
			each chapter & Usage reviewed and decisions made as follows. Literature review
			chapters (2, and 3) are ended with a Summary section summarising the findings given in the
			literature. Research content (Chapters 4, 5, 6 and Sections 7.2 and 7.3) are concluded with a
			Conclusion section. Chapter 7 then ends with a Summary section to summarise the finding of the two
			experiments in that chapter. To keep this patter the title of Chapter~6 has been
			changed from ``Conclusions'' to ``Conclusion''. \tabularnewline
		\hline
		31. & Provide a new Appendix to list software contributions and tools used & \tabularnewline
		\hline
		32. & Carry out a final proof read of the revised text & \tabularnewline
		\hline
	\end{longtable}
\end{center}
\end{document}
