%Review of timbral control research.
%	Timbre Definition
%	Timbre Spaces (MDS and shit)
%	Audio Features
%	Perceptual Control of Synthesis

\chapter{Timbre}
\label{chap:Timbre}
	There are three properties which describe how a sound is perceived, these being loudness, pitch and timbre. Loudness
	describes the perceived intensity of a sound and pitch its perceived frequency. Timbre then describes any other
	properties of a sound, besides loudness and pitch, which allow it to be distinguished from other sounds
	\citep{mathews1999introduction}. Loudness and pitch are both one dimensional properties allowing sounds to be
	ordered from quiet to loud or low to high pitch. Timbre is a more complex property consisting of multiple dimensions
	\citep{rossing2002the}. There is a large body of research concerning the analysis of timbre, identifying these
	dimensions and their relationships with the acoustic features of a sound.

	Simple descriptions of a sounds timbre involve instrument identification. A sound could be described as `cello-like'
	or `flute-like'. More broadly the class of instrument, string or woodwind, could be used to describe the timbre of a
	sound. While these terms are useful for discussing the instrumentation of pieces they can not be applied generally
	to a wide range of timbres. It is not very useful to describe the timbre of a xylophone as being `not flute-like'.

	More general timbral descriptors directly describe the sound itself rather than the source which produced it. These
	include terms such as bright, rough and sharp. This allows the timbre of different sounds to be compared according
	to these terms \citep{howard2009acoustics}. Sounds can also be ordered in respect to these criteria much like with
	loudness and pitch. For example one could order a set of sounds by how bright they sound.

	Early research into timbre was performed by \citet{helmholtz1875on}. More recent work involves research from various
	fields. Low level features of audio segments can be found using signal analysis techniques. More complicated
	information about the perception of a signal can be discovered through modelling the behaviour of the human hearing
	system. Lastly experiments can be undertaken in which participants listen to audio samples and provide responses
	regarding the timbre of the samples. These responses are then analysed to uncover any correlations between the
	participants responses and lower level features of the audio samples.

	This chapter will review the existing body of timbral research. Section \ref{sec:Timbre-LowLevelFeatures} discusses
	metric which are used to describe the low level features of audio signals. Section
	\ref{sec:Timbre-PsychoacousticPrinciples} covers various models which describe the perception of various auditory
	phenomena. \note{Put in what the rest of the sections are about}.

\section{Low Level Audio Features}
\label{sec:Timbre-LowLevelFeatures}
	A widely cited definition of timbre \citep{ASA1960american} suggests that timbre in influenced by various low level
	features of an audio signal. The spectral content, waveform and temporal characteristics all effect the perceived
	timbre of a sound. Signal analysis techniques can be used to extract information about these elements of a signal.
	A large list of such features feature extraction techniques is given by \citep{peeters2004a}. These features can be
	separated into three categories. Features which describe the properties of a signals waveform and how it evolves
	with time (temporal features), features which describe the frequency content of a signal (spectral features) and
	features which describe how the frequency content of a signal evolves with time (spectro-temporal features).

	\subsection{Temporal Features}
	\label{sec:Timbre-LowLevelFeatures-Temporal}
		Simple temporal features involve taking statistical measurements, such as mean and variance, of the audio
		samples in a signal. 

		\note
		{
			Envelope Detection (ADSR).
		}

	\subsection{Spectral Features}
	\label{sec:Timbre-LowLevelFeatures-Spectral}
		\note
		{
			\begin{itemize}
				\item Spectral statistics.
				\item Spectral shape.
				\item MFCCs.
			\end{itemize}

			MFCCs originally developed for speech recognition \citep{davis1980comparison} but used for other
			timbral tasks more recently \citep{depoli1997sonological}.
		}

	\subsection{Spectro-Temporal Features}
	\label{sec:Timbre-LowLevelFeatures-Spectrotemporal}
		\note
		{
			Delta MFCC and spectral flux and such and the like.
		}

	\citet{howard2009acoustics} separates sounds into three separate sections. The steady state section describes the
	middle portion of a sound during which the timbre does not vary much with time. The onset and offset sections
	describe how the sound evolves when rising from silence to the steady state and falling back to silence after the
	steady state. These sections together describe the envelope of the sound. It is proposed that the onset and steady
	state sections influence timbre to a larger degree than the offset section.

	\note
	{
		Stuff like this also \citep{freed1990auditory, lakatos2000a}.
	}
	
\section{Psychoacoustic Principles}
\label{sec:Timbre-PsychoacousticPrinciples}
	Psychoacoustics is a field which deals with the perception of sound. The existing literature concerns the study of
	the human hearing system and how it responds to certain aspects of audio stimuli. Several different areas of audio
	perception have been researched. Methods have been devised to model the human perception of loudness
	\citep{moore1997a} and pitch \citep{gerhard2003pitch}. Other research considers the human hearing systems ability to
	locate sound sources \citep{blauert1997spatial}. This section will summarise the psychoacoustic principles which are
	useful for investigating the perception of timbre. 

	\note
	{
		Critical bands and masking.

		Resolving harmonics in the cochlea. Low order harmonics are generally resolved individually so their
		individual levels will have a strong effect on perceived timbre. Higher order harmonics are resolved in
		groups so less manipulation is needed \citep{howard2009acoustics}.
	}

\section{Timbral Features}
\label{sec:Timbre-TimbralFeatures}
	\note
	{
		Psychoacoustic principles have been used to develop timbral metrics (sharpness, roughness).
	}

\section{Parameterisation of Timbre}
\label{sec:Timbre-Parameterisation}
	\note{Analysis of timbre spaces and discussion on salience of some features as given in the literature.}

	\subsection{Dissimilarity Tests}
	\label{sec:Timbre-Dissimilarity}
		\note
		{
			A more traditional approach, \citet{grey1977multidimensional} and all the copycat lot 
			\citep{burgoyne2008a, caclin2005acoustic}.
		}

	\subsection{VAME}
	\label{sec:Timbre-VAME}
		\note{\citet{kendall1993verbal1, kendall1993verbal2} trying to upset the status quo.}

	\subsection{Parameter Space}
	\label{sec:Timbre-ParameterSpaces}
	\note{Research like Social EQ and stuff \citep{cartwright2013socialeq, seetharaman2014crowdsourcing}.}

\section{Controlling Timbre}
\label{sec:Timbre-Control}
	\note{Lots of them synthesis dudes have tried to control the beast. \citet{zacharakis2011an} did some stuff.}
	
	\note
	{
		Some dudes do morphing (analysis resynthesis from what I remember) like good old 
		\citet{williams2007perceptually, williams2009perceptually, williams2010perceptually}
	}
