%Review of timbral control research.
%	Timbre Definition
%	Timbre Spaces (MDS and shit)
%	Audio Features
%	Perceptual Control of Synthesis

\chapter{Timbre}
\label{chap:Timbre}
	There are three properties which describe how a sound is perceived, these being loudness, pitch and timbre. Loudness
	describes the perceived intensity of a sound and pitch its perceived frequency. Timbre then describes any other
	properties of a sound, besides loudness and pitch, which allow it to be distinguished from other sounds
	\citep{mathews1999introduction}. Loudness and pitch are both one dimensional properties allowing sounds to be
	ordered from quiet to loud or low to high pitch. Timbre is a more complex property consisting of multiple dimensions
	\citep{rossing2002the}. There is a large body of research concerning the analysis of timbre, identifying these
	dimensions and their relationships with the acoustic features of a sound.

	Simple descriptions of a sounds timbre involve instrument identification. A sound could be described as `cello-like'
	or `flute-like'. More broadly the class of instrument, string or woodwind, could be used to describe the timbre of a
	sound. While these terms are useful for discussing the instrumentation of pieces they can not be applied generally
	to a wide range of timbres. It is not very useful to describe the timbre of a xylophone as being `not flute-like'.

	More general timbral descriptors directly describe the sound itself rather than the source which produced it. These
	include terms such as bright, rough and sharp. This allows the timbre of different sounds to be compared according
	to these terms \citep{howard2009acoustics}. Sounds can also be ordered in respect to these criteria much like with
	loudness and pitch. For example one could order a set of sounds by how bright they sound.

	The study of timbre involves work from various fields. Low level features of audio segments can be found using
	signal analysis techniques. More complicated information about the perception of a signal can be discovered through
	modelling the behaviour of the human hearing system. Lastly experiments can be undertaken in which participants
	listen to audio samples and provide responses regarding the timbre of the samples. These responses are then analysed
	to uncover any correlations between the participants responses and lower level features of the audio samples.

	This chapter will review the existing body of timbral research. Starting with the analysis of signal features,
	through the human perception of sound, to the development of experimental methodologies for assessing timbre.

\section{Low Level Audio Features}
\label{sec:Timbre-Features}
	\note
	{
		Bits and bobs about the audio features used in MIR and stuff \citep{peeters2004a}.
		
		Stuff like this also \citep{freed1990auditory, lakatos2000a}.
	}

\section{Psychoacoustics}
\label{sec:Timbre-Psychoacoustics}
	\note
	{
		Resolving harmonics in the cochlea. Low order harmonics are generally resolved individually so their
		individual levels will have a strong effect on perceived timbre. Higher order harmonics are resolved in
		groups so less manipulation is needed \citep{howard2009acoustics}.
	}

\section{Parameterisation of Timbre}
\label{sec:Timbre-Parameterisation}
	\note{Analysis of timbre spaces and discussion on salience of some features as given in the literature.}

	\subsection{Dissimilarity Tests}
	\label{sec:Timbre-Dissimilarity}
		\note
		{
			A more traditional approach, \citet{grey1977multidimensional} and all the copycat lot 
			\citep{burgoyne2008a, caclin2005acoustic}.
		}

	\subsection{VAME}
	\label{sec:Timbre-VAME}
		\note{\citet{kendall1993verbal1, kendall1993verbal2} trying to upset the status quo.}

	\subsection{Parameter Space}
	\label{sec:Timbre-ParameterSpaces}
	\note{Research like Social EQ and stuff \citep{cartwright2013socialeq, seetharaman2014crowdsourcing}.}

\section{Controlling Timbre}
\label{sec:Timbre-Control}
	\note{Lots of them synthesis dudes have tried to control the beast. \citet{zacharakis2011an} did some stuff.}
	
	\note
	{
		Some dudes do morphing (analysis resynthesis from what I remember) like good old 
		\citet{williams2007perceptually, williams2009perceptually, williams2010perceptually}
	}
