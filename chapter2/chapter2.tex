%Review of timbral control research.
%	Timbre Definition
%	Timbre Spaces (MDS and shit)
%	Audio Features
%	Perceptual Control of Synthesis

\chapter{Timbre}
\label{chap:Timbre}
	There are three properties which describe how a sound is perceived, these being loudness, pitch and timbre. Loudness
	describes the perceived intensity of a sound and pitch its perceived frequency. Timbre then describes any other
	properties of a sound, besides loudness and pitch, which allow it to be distinguished from other sounds
	\citep{mathews1999introduction}. Loudness and pitch are both one dimensional properties allowing sounds to be
	ordered from quiet to loud or low to high pitch. Timbre is a more complex property consisting of multiple
	dimensions. There is a large body of research concerning the analysis of timbre, identifying these dimensions and
	their relationships with the acoustic features of a sound.

	

\section{Defining Timbre}
\label{sec:Timbre-Definition}
	\note{What is timbre?}

\section{Psychoacoustics}
	\note
	{
		Resolving harmonics in the cochlea. Low order harmonics are generally resolved individually so their
		individual levels will have a strong effect on perceived timbre. Higher order harmonics are resolved in
		groups so less manipulation is needed \citep{howard2009acoustics}.
	}

\section{Audio Features}
\label{sec:Timbre-Features}
	\note
	{
		Bits and bobs about the audio features used in MIR and stuff \citep{peeters2004a}.
		
		Stuff like this also \citep{freed1990auditory, lakatos2000a}.
	}

\section{Parameterisation of Timbre}
\label{sec:Timbre-Parameterisation}
	\note{Analysis of timbre spaces and discussion on salience of some features as given in the literature.}

	\subsection{Dissimilarity Tests}
	\label{sec:Timbre-Dissimilarity}
		\note
		{
			A more traditional approach, \citet{grey1977multidimensional} and all the copycat lot 
			\citep{burgoyne2008a, caclin2005acoustic}.
		}

	\subsection{VAME}
	\label{sec:Timbre-VAME}
		\note{\citet{kendall1993verbal1, kendall1993verbal2} trying to upset the status quo.}

	\subsection{Parameter Space}
	\label{sec:Timbre-ParameterSpaces}
	\note{Research like Social EQ and stuff \citep{cartwright2013socialeq, seetharaman2014crowdsourcing}.}

\section{Controlling Timbre}
\label{sec:Timbre-Control}
	\note{Lots of them synthesis dudes have tried to control the beast. \citet{zacharakis2011an} did some stuff.}
	
	\note
	{
		Some dudes do morphing (analysis resynthesis from what I remember) like good old 
		\citet{williams2007perceptually, williams2009perceptually, williams2010perceptually}
	}
