\documentclass[a4paper]{article}

\usepackage[cm]{fullpage}
\renewcommand{\familydefault}{\sfdefault}

\title{Sean's Thesis Story}
\author{}
\date{}

% Once upon a time there was a boy called Sean. He wrote a thesis, the end.

\begin{document}
\maketitle
	
	\subsubsection*{Chapter 1}
	We begin with the usual chapter about scope and motivation and such and the like.

	\subsubsection*{Chapter 2}
	This chapter will be dedicated to the review of research into timbre. Including discussion of:

		\begin{itemize}
			\item Audio features.
			\item Listening test methodologies.
			\item Analysis techniques (MDS and so forth).
			\item Previously developed timbral control techniques.
		\end{itemize}

		Previous research has looked into using traditional audio effects to apply perceptual transformations. I am
		going to focus on doing this type of thing with distortion (harmonic excitation).This then leads on into
		chapter 3.

	\subsubsection*{Chapter 3}
	This chapter will focus on the various methods for analysing and applying distortion. Covering:

		\begin{itemize}
			\item Some background on distortion / nonlinear systems analysis.
			\item Timbral aspects of nonlinear distortion.
			\item Definition of harmonic excitation.
			\item Uses of harmonic excitation.
			\item Harmonic excitation methods, including those from the SMC paper.
		\end{itemize}

	\subsubsection*{Chapter 4}
	This chapter will combine the work in the previous two chapters. The main focus being the analysis of how well the
	harmonic excitation methods can be applied to the problem of timbral control. 
	
	Beginning with a discussion of desirable properties effects should have in order to give good timbral control. This
	leads to the development of a methodology to assess the suitability of given harmonic exciters for use in timbral
	control. This methodology can then be used to asses the algorithms from the previous chapter.

	The assessment of the algorithms will produce information on specific audio features which can be controlled by
	harmonic exciters. Now we have an ability to control specific audio features so we need to know how these features
	map to perceptual attributes (bring on chapter 5).

	\subsubsection*{Chapter 5}
	This chapter will cover the experiments undertaken to gather mappings from audio features to perceptual
	descriptors. Some well know methodologies (VAME, Mushra) were evaluated but did not provide the necessary
	information.

	For this reason the SAFE methodology was developed. And so there is a description of the SAFE project and its
	proposed benefits.

	\subsubsection*{Chapter 6}
	This chapter brings together the data from chapters 4 and 5. With the data from SAFE (and maybe some other places)
	we can see which descriptors are controlled by the same features that the previously discussed algorithms can
	control. This can lead to the development of mappings between low level effect parameters and semantic features for
	this type of effect.

	\subsubsection*{Chapter 7}
	Finally we can produce an effect (or multiple) which use the developed mappings and evaluate it with some user
	tests. This will be a nice validation of the research's success (or failure :P).

\end{document}
