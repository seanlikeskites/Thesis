\documentclass[a4paper]{article}

\usepackage[cm]{fullpage}
\renewcommand{\familydefault}{\sfdefault}

\title{Story}
\author{}
\date{}

% Once upon a time there was a boy called Sean. He wrote a thesis, the end.

\begin{document}
\maketitle

	We begin with the usual stuff about scope and motivation and such and the like.

	Introduction to timbre and research in that area.

	One class of effects that is used to shape timbre is distortion. Starting with an overview of nonlinear systems theory (wow that is far too grandiose a title), moving into more and more specialised `exciter' like technology for more advanced control. We can discuss implementation details, uses and timbre of these effects.

	One downside of nonlinear systems the inability to write general expressions for the spectral transformations they will apply to signals. This can make them difficult to use as general `perceptual control' effects. We can define some assessment methodology to grade how broadly a given system can apply. We can then discuss the most salient audio feature which can be adjusted by these effects.

	In order to map these feature transformations to semantic descriptors, listening tests were undergone. VAME and MUSHRA style tests were tried but to no avail (perhaps not suitable or maybe too low a number of participants). SAFE was developed to gather a large amount of descriptors and feature data from a wide audience.

	With the data from SAFE (and maybe some other places) we can see which descriptors are controlled by the same features that the previously discussed algorithms can control. This can lead to the development of mappings between low level effect parameters and semantic features for this type of effect.

	Finally we can produce an effect (or multiple) which use the developed mappings and evaluate it with some user tests. This will be a nice validation of the research's success (or failure :P).

\end{document}
