\chapter{Perceptual Listening Test Methodologies}
\label{chap:SAFE}

\section{Laboratory Listening Tests}
\label{sec:Laboratory-Listening-Tests}

\subsection{Multiple Stimuli Tests}
\label{sec:Multiple-Stimuli-Tests}
	In a multiple stimuli test participants are presented with several stimuli at ones and asked to rate each again a
	given criteria. A multiple stimuli methodology often used in perceptual audio tests is MUSHRA (Multiple Stimuli with
	Hidden Reference and Anchor) \citep{mushra2014}.

\subsection{VAME}
\label{sec:VAME}
\citet{kendall1993verbal1, kendall1993verbal2}

\section{Distributed Listening Tests}
\label{sec:Distributed-Listening-Tests}
	More recent timbral research has attempted to gather information from a much larger group of participant at the loss
	of control over listening environment.

\subsection{Social EQ and Reverb}
\label{sec:Social-EQ-and-Reverb}
	\citet{cartwright2013socialeq} and \citet{seetharaman2014crowdsourcing} used online applications in order to collect
	information about the semantic descriptors of equalisation and reverb.

\subsection{Production Timbre Assessment}
\label{sec:Production-Timbre-Assessment}
