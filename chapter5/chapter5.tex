\chapter{Perceptual Listening Test Methodologies}
\label{chap:SAFE}

\section{Laboratory Listening Tests}
\label{sec:Laboratory-Listening-Tests}

\subsection{Multiple Stimuli Tests}
\label{sec:Multiple-Stimuli-Tests}
	In a multiple stimuli test participants are presented with several stimuli at ones and asked to rate each again a
	given criteria. A multiple stimuli methodology often used in perceptual audio tests is MUSHRA (Multiple Stimuli with
	Hidden Reference and Anchor) \citep{mushra2014}.

\subsection{VAME}
\label{sec:VAME}
\citet{kendall1993verbal1, kendall1993verbal2}

\section{Distributed Listening Tests}
\label{sec:Distributed-Listening-Tests}
	More recent timbral research has attempted to gather information from a much larger group of participants at the
	loss of control over listening environment.

\subsection{Social EQ and Reverb}
\label{sec:Social-EQ-and-Reverb}
	\citet{cartwright2013socialeq} and \citet{seetharaman2014crowdsourcing} used online applications in order to collect
	information about the semantic descriptors of equalisation and reverb.

\subsection{Production Timbre Assessment} % this name will probably change
\label{sec:Production-Timbre-Assessment}
	The previously discussed testing methodologies all rely on the participants performing a certain set of tasks. While
	this structure helps to reduce the number of variables in an experiment it does not necessarily reflect the way
	audio is treated in a production environment.

	A new methodology has been developed in which the analysis of timbre is introduced into a typical music production
	workflow causing minimal interruption to the producer. This section will detail what the typical production workflow
	is and how semantic information can be gathered.

\subsubsection{Music Production Workflow}
\label{sec:Music-Production-Workflow}
	\todo{Find some references for this section. Probably mixing engineers handbook or something.}

	The music production workflow has four main stages:

	\begin{itemize}
		\item Recording
		\item Editing
		\item Mixing
		\item Mastering
	\end{itemize}

	At every stage of this process semantic descriptors are often used to communicate the desired timbral qualities of
	the audio. For instance one my ask that a certain microphone be used because of the `warmth' it adds to the recorded
	sound. During the mixing and mastering stages audio processing effects are applied to shape the timbre further.
	These stages will be the focus of this section as the aim of this thesis is to improve the intuitiveness of these
	effects.

	Historically audio effects were pieces of electronic hardware through which an audio signal is passed. Modern music
	production techniques utilise Digital Audio Workstation (DAW) software. This software enables users to record, edit
	and mix multiple tracks of audio using a computer. 
