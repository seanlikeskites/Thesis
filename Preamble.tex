\maketitle

%\cleardoublepage
%\thispagestyle{plain}
%\vspace*{\fill}
%\begin{center}
%	\includegraphics[width=0.8\textwidth]{Images/seanFace.jpg}
%	\newline
%	\newline
%	\Huge
%	Welcome to my thesis, are your ready for some fun?
%	\normalsize
%\end{center}
%\vspace*{\fill}

\begin{abstract}
	Timbre is the property of a sound which allows it to be distinguished from other sounds of the same loudness and
	pitch. These differences can also be demonstrated mathematically as using various low level signal features.
	Research into timbre typically focuses on analysing the multidimensional relationship between these low level
	features and the perceived timbre. In music production the timbre of recorded sounds is altered through the
	application of audio effects. A commonly used class of audio effects is harmonic excitation, in which new spectral
	content is introduced to the signal at harmonic frequencies. This thesis focuses on the uses of harmonic excitation
	effects for the control of timbre. For a harmonic excitation system to provide control over the timbre of a signal
	it must be able to manipulate the associated low level features. As such, various harmonic excitation algorithms are
	evaluated against a set of criteria which describe how well they are suited to applying timbral control and it was
	found that the most flexibility was provided by methods which allow for the generation of individual harmonics.
	Techniques are then developed by which these algorithms can be used to provide monotonic control over various low
	level features of audio signals. 

	In music production these timbral manipulations are often described using semantic descriptors (e.g. `warm' and
	`bright').  The timbral meanings of these terms can be discerned through analysis of the relationship between their
	use to describe audio signals and the signal's low level features. To investigate these relationships a study is
	conducted analysing the changes to perceived timbre and audio features associated with the application of distortion
	and equalisation effects. Four distinct timbral groups are identified (warmth, brightness, crunchiness and
	muddiness) each consisting of related semantic terms. The perceptual meanings of these semantic groups are uncovered
	by measuring how the low level features of signals are manipulated by the application of the effects. The
	distinction between each of the groups are found to be highly correlated with several low level audio features.
	
	Significant experience is required to be able to utilise traditional studio equipment to manipulate the timbre of a
	sound based on a semantic description. Recently, new tools have been developed which have control parameters
	labelled with commonly used timbral descriptors, reducing the amount of experience needed. These systems manipulate
	the low level features of a signal associated with a particular timbral group. In this work two such systems are
	proposed utilising harmonic excitation algorithms: one moving audio between the warmth and brightness timbral groups
	and the other between the brightness and crunchiness groups. Subjective listening tests are conducted to evaluate
	the performance of these systems. Both are successful in processing audio such that is is perceived as part of the
	brightness group, participants exhibiting a minimum accuracy of 72\% in identifying the timbral group. When
	configured to introduce warmth or crunchiness the systems perform less well, participants only identifying the
	correct timbral group 54\% or 51\% of the time respectively.

	\note{mention reconstruction}
\end{abstract}

\begin{acknowledgements}
	\note
	{
		Thanks guys!
	}

	\begin{center}
		\includegraphics[width=0.3\textwidth]{Images/orca.pdf}
	\end{center}
\end{acknowledgements}

\tableofcontents
\listoffigures
\listoftables
\listof{datum}{List of Data}
\printglossaries
\cleardoublepage
