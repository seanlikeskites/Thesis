\maketitle

%\begin{center}
%	\includegraphics[width=0.8\textwidth]{Images/seanFace.jpg}
%	\newline
%	\newline
%	\Huge
%	Welcome to my thesis, are your ready for some fun?
%	\normalsize
%\end{center}

\begin{abstract} 
	In music production it is often necessary to discuss creative decisions using semantic terms which refer to the
	timbral features of audio segments. Significant experience is required to be able to utilise traditional studio
	equipment to manipulate the timbre of a sound based on a semantic description. Recently, new tools have been
	developed which have control parameters labelled with commonly used timbral descriptors, reducing the amount of
	experience needed. This thesis explores the application of harmonic excitation effects in the development of
	these `semantically controlled' audio effects'.

	Firstly, a study is conducted to investigate the language used to describe the application of equalisation and
	distortion effects. Four distinct timbral groups are identified (warmth, brightness, crunchiness and muddiness) each
	consisting of related semantic terms. The meanings of these semantic groups are uncovered by measuring how low level
	features of audio signals are manipulated by the application of the effects. Salient features are identified which
	describe the differences between the timbral groups and, to a lesser extent, the differences between descriptors in
	a particular group.
	
	Secondly, various nonlinear and time variant algorithm are evaluated against a set of criteria which describe how
	well they suited to applying timbral control. Techniques are then developed by which they can be used to
	successfully manipulate specific low level features of audio signals. Using these techniques two systems are
	proposed for the control of timbre within and between timbral groups, one moving audio between warmth and brightness
	and the other between brightness and crunchiness. Subjective listening tests are conducted to evaluate the
	performance of these systems, finding that both are successful in processing audio such that is is perceived as part
	of the brightness group. When configured to introduce warmth or crunchiness the systems perform less well,
	participants often describing their effects using terms from outside the relevant timbral group.
\end{abstract}

\tableofcontents
\listoffigures
\listoftables
\listof{datum}{List of Data}
