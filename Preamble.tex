\maketitle

%\begin{center}
%	\includegraphics[width=0.8\textwidth]{Images/seanFace.jpg}
%	\newline
%	\newline
%	\Huge
%	Welcome to my thesis, are your ready for some fun?
%	\normalsize
%\end{center}

\begin{abstract} 
	In music production it is often necessary to discuss creative decisions using semantic terms which refer to the
	timbral features of audio segments. Significant experience is required to be able to utilise traditional studio
	equipment to manipulate the timbre of a sound based on a semantic description. Recently, new tools have been
	developed which have control parameters labelled with commonly used timbral descriptors, reducing the amount of
	experience needed. This thesis explores the application of harmonic excitation effects in the development of these
	`semantically controlled' audio effects, harmonic excitation being the deliberate addition of new spectral content
	to a signal, particularly at harmonic frequencies.

	Perceived timbre is dependant on various low level features of an audio signal. To investigate this relationship a
	study is conducted analysing the changes to perceived timbre and audio features associated with the application of
	distortion and equalisation effects. Four distinct timbral groups are identified (warmth, brightness, crunchiness
	and muddiness) each consisting of related semantic terms. The meanings of these semantic groups are uncovered by
	measuring how the low level features of signals are manipulated by the application of the effects. The distinction
	between warmth and brightness correlates highly ($r = 0.867$, $p < 0.01$) with the changes in spectral centroid,
	while the crunchiness group is best identified by changes in spectral irregularity ($r = 0.992$, $p < 0.01$).
	
	To provide control over these timbral descriptors an effect must be able to replicate the associated feature
	changes. Various nonlinear and time variant algorithms are evaluated against a set of criteria which describe how
	well they suited to applying timbral control, methods which allow for the generation of individual harmonics being
	shown to provide the most flexibility. Techniques are then developed by which these algorithms can be used to
	provide monotonic control over particular low level features of audio signals. Using these techniques two systems
	are proposed for the control of timbre within and between timbral groups, one moving audio between warmth and
	brightness and the other between brightness and crunchiness. Subjective listening tests are conducted to evaluate
	the performance of these systems. Both are successful in processing audio such that is is perceived as part of the
	brightness group, participants exhibiting a minimum accuracy of 72\% in identifying the timbral group. When
	configured to introduce warmth or crunchiness the systems perform less well, participants only identifying the
	correct timbral group 54\% or 51\% of the time respectively.
\end{abstract}

\tableofcontents
\listoffigures
\listoftables
\listof{datum}{List of Data}
