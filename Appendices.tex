\begin{appendices}
\chapter{Mathematical Derivations}
\label{app:MathematicalDerivations}
	\section{Fourier Series of a Full Wave Rectified Sinusoid}
	\label{app:MathematicalDerivations-Rectification}
		The complex Fourier series coefficients, $c_{n}$ of a full wave rectified sinusoid can be calculated as
		follows:

		\[ c_{n} = \frac{1}{2\pi} \int_{-\pi}^{\pi} |\sin(x)|e^{-inx} dx \]

		Expanding out the complex exponential gives:

		\[ c_{n} = \frac{1}{2\pi} \int_{-\pi}^{\pi} \left( |\sin(x)|\cos(nx) - i|\sin(x)|\sin(nx) \right) dx \]

		The expression $|\sin(x)|\sin(nx)$ is an odd function, so across the interval of our Fourier series integral
		we have:

		\[ \int_{-\pi}^{\pi} i|\sin(x)|\sin(nx) = 0 \]

		Removing this term from the integral leaves us with:

		\[ c_{n} = \frac{1}{2\pi} \int_{-\pi}^{\pi} |\sin(x)|\cos(nx) dx \]

		The expression $|\sin(x)|\cos(nx)$ is an even function, so we can adjust the interval over which we take the
		integral:

		\begin{align}
			c_{n} & = \frac{1}{\pi} \int_{0}^{\pi} \sin(x)\cos(nx) dx \nonumber \\
			\nonumber \\
			& = \frac{1}{2\pi} \int_{0}^{\pi} \sin((n+1)x) + \sin((1-n)x) dx \nonumber \\
			\nonumber \\
			& = \frac{1}{2\pi} \left[ -\frac{\cos((n+1)x)}{n+1} - 
				\frac{\cos((1-n)x)}{1-n} \right]_{0}^{\pi} \nonumber
		\end{align}

		When $n$ is even:

		\begin{align}
			c_{n} & = \frac{1}{2\pi} \left( \frac{1}{n+1} + \frac{1}{1-n} 
				+ \frac{1}{n+1} + \frac{1}{1-n} \right) \nonumber \\
			\nonumber \\
			& = \frac{1}{2\pi} \frac{2(1-n) + 2(n+1)}{1 - n^{2}} \nonumber \\
			\nonumber \\
			& = \frac{2}{\pi(1 - n^{2})} \nonumber
		\end{align}

		When $n$ is odd:

		\begin{align}
			c_{n} & = \frac{1}{2\pi} \left( -\frac{1}{n+1} - 
				\frac{1}{1-n} + \frac{1}{n+1} + \frac{1}{1-n} \right) \nonumber \\
			\nonumber \\
			& = 0 \nonumber
		\end{align}

		Giving:

		\[ c_{n} = \begin{cases}
				\frac{2}{\pi(1 - n^{2})} & \text{when $n$ is even} \\
				0 & \text{when $n$ is odd}
			\end{cases} \]

	\section{Fourier Series of a Sinusoid after Application of an Integrator}
	\label{app:MathematicalDerivations-Integrator}
		Applying the integrator in Equation~\ref{eq:Integrator} to a sinusoid $\sin(x)$, sampled at a rate of
		$f_{s}$, produces a periodic waveform, $y$, with a period of $2\pi$ defined on the interval $0$ to $2\pi$
		as:

		\[ y = \begin{cases}
				kf_{s} (1 - \cos(x)) & \text{when $0 \leq x < \pi$} \\
				kf_{s} (3 + \cos(x)) & \text{when $\pi \leq x < 2\pi$} \\
			\end{cases} \]

		The complex Fourier series coefficients, $c_{n}$, of this waveform can be calculated as:

		\[ c_{n} = \frac{kf_{s}}{2\pi} \left( \int_{0}^{\pi} (1-\cos(x))e^{-inx} dx
						      + \int_{\pi}^{2\pi} (3 + \cos(x))e^{-inx} dx \right) \]

		Evaluating the first integral gives:
		
		\begin{align}
			\int_{0}^{\pi} (1-\cos(x))e^{-inx} dx & = 
				\int_{0}^{\pi} \left( e^{-inx} - \cos(x)e^{-inx} \right) dx \nonumber \\
			\nonumber \\
			& = \left[ \frac{ie^{-inx}}{n} - \frac{e^{-inx}(\sin(x) 
				- in\cos(x))}{1 - n^{2}} \right]_{0}^{\pi} \nonumber \\
			\nonumber \\
			& = \frac{ie^{-in\pi}}{n} - \frac{ine^{-in\pi}}{1 - n^{2}} 
				- \frac{i}{n} - \frac{in}{1 - n^{2}} \nonumber \\
			\nonumber \\
			& = \frac{i}{n(1-n^{2})} \left( (1 - n^{2})e^{-in\pi} - 
				n^{2}e^{-in\pi} - (1 - n^{2}) - n^2 \right) \nonumber \\
			\nonumber \\
			& = \frac{i}{n - n^{3}} \left( e^{-in\pi} - 2n^{2}e^{-in\pi} - 1 \right) \nonumber
		\end{align}

		And the second integral yields:

		\begin{align}
			\int_{\pi}^{2\pi} (3+\cos(x))e^{-inx} dx & = 
				\int_{\pi}^{2\pi} \left( 3e^{-inx} + \cos(x)e^{-inx} \right) dx \nonumber \\
			\nonumber \\
			& = \left[ \frac{3ie^{-inx}}{n} + \frac{e^{-inx}(\sin(x) - 
				in\cos(x))}{1 - n^{2}} \right]_{\pi}^{2\pi} \nonumber \\
			\nonumber \\
			& = \frac{3i}{n} - \frac{in}{1 - n^{2}} - \frac{3ie^{-in\pi}}{n} - 
				\frac{ine^{-in\pi}}{1 - n^{2}} \nonumber \\
			\nonumber \\
			& = \frac{i}{n(1-n^{2})} \left( 3(1-n^{2}) - n^{2} - 
				3(1-n^{2})e^{-in\pi} - n^{2}e^{-in\pi} \right) \nonumber \\
			\nonumber \\
			& = \frac{i}{n - n^{3}} \left( 2n^{2}e^{-in\pi} - 3e^{-in\pi} - 4n^{2} + 3 \right) \nonumber
		\end{align}

		Combining these results gives:

		\begin{align}
			c_{n} & = \frac{kf_{s}}{2\pi} \times \frac{i}{n - n^{3}} \left( 2 - 2e^{-in\pi} - 4n^{2} \right) 
				\nonumber \\
			\nonumber \\
			& = \frac{ikf_{s}}{\pi} \left( \frac{2n^{2} + e^{-in\pi} - 1}{n^{3} - n} \right) \nonumber
		\end{align}

		When $n = 1$ this will involve a division by 0, so we can find $c_{1}$ as:

		\[ c_{1} = \frac{kf_{s}}{2\pi} \left( \int_{0}^{\pi} (1-\cos(x))e^{-ix} dx
						      + \int_{\pi}^{2\pi} (3 + \cos(x))e^{-ix} dx 
						      \right) \]

		Starting with the first integral:
		
		\begin{align}
			\int_{0}^{\pi} (1-\cos(x))e^{-ix} dx & = \int_{0}^{\pi} e^{-ix} - 
				\cos^{2}(x) + i\cos(x)\sin(x) dx \nonumber \\
			\nonumber \\
			& = \int_{0}^{\pi} e^{-ix} - \frac{1}{2} - 
				\frac{\cos(2x)}{2} + \frac{i\sin(2x)}{2} dx \nonumber \\
			\nonumber \\
			& = \left[ ie^{-ix} - \frac{x}{2} - \frac{\sin(2x)}{4} - 
				\frac{i\cos(2x)}{4} \right]_{0}^{\pi} \nonumber \\
			\nonumber \\
			& = -i - \frac{\pi}{2} - \frac{i}{4} - \left( i - \frac{i}{4} \right) \nonumber \\
			\nonumber \\
			& = -2i - \frac{\pi}{2} \nonumber
		\end{align}

		And the second:
		
		\begin{align}
			\int_{\pi}^{2\pi} (3+\cos(x))e^{-ix} dx & = 
				\int_{\pi}^{2\pi} 3e^{-ix} + \cos^{2}(x) - i\cos(x)\sin(x) dx \nonumber \\
			\nonumber \\
			& = \int_{\pi}^{2\pi} 3e^{-ix} + \frac{1}{2} + \frac{\cos(2x)}{2} - 
				\frac{i\sin(2x)}{2} dx \nonumber \\
			\nonumber \\
			& = \left[ 3ie^{-ix} + \frac{x}{2} + \frac{\sin(2x)}{4} + 
				\frac{i\cos(2x)}{4} \right]_{\pi}^{2\pi} \nonumber \\
			\nonumber \\
			& = 3i + \pi + \frac{i}{4} - \left( -3i + \frac{\pi}{2} + \frac{i}{4} \right) \nonumber \\
			\nonumber \\
			& = 6i + \frac{\pi}{2} \nonumber
		\end{align}

		Which combine to give:

		\[ c_{1} = \frac{2ikf_{s}}{\pi} \]

		Which also allows us to write:

		\[ c_{-1} = -\frac{2ikf_{s}}{\pi} \]

		Finally, when $n = 0$ division by 0 will also occur. $c_{0}$ is calculated as follows:

		\[ c_{0} = \frac{kf_{s}}{2\pi} \left( \int_{0}^{\pi} 1-\cos(x) dx
						      + \int_{\pi}^{2\pi} 3 + \cos(x) dx 
						      \right) \]

		Starting with the first integral:

		\begin{align}
			\int_{0}^{\pi} 1 - \cos(x) dx & = \biggl[ x + \sin(x) \biggr]_{0}^{\pi} \nonumber \\
			\nonumber \\
			& = \pi \nonumber
		\end{align}

		And the second:

		\begin{align}
			\int_{\pi}^{2\pi} 3 + \cos(x) dx & = \biggl[ 3x - \sin(x) \biggr]_{\pi}^{2\pi} \nonumber \\
			\nonumber \\
			& = 6\pi - 3\pi \nonumber \\
			\nonumber \\
			& = 3\pi \nonumber
		\end{align}

		Leading to:

		\[ c_{0} = 2kf_{s} \]

		Which gives the Fourier series coefficients as:

		\begin{gather}
			c_{-1} = - \frac{2ikf_{s}}{\pi} \nonumber \\
			c_{0} = 2kf_{s} \nonumber \\
			c_{1} = \frac{2ikf_{s}}{\pi} \nonumber \\
			c_{n} = \frac{ikf_{s}}{\pi} \left( \frac{2n^{2} + e^{-in\pi} - 1}{n^{3} - n} \right) \nonumber
		\end{gather}

	\section{Parameterisation of Spectral Flatness}
	\label{app:MathematicalDerivations-SpectralFlatness}
		Referring back to Section~\ref{sec:FeatureControl-Parameterisation-Flatness} we recall that the spectral
		flatness was manipulated by altering the arithmetic mean of the spectral components' amplitudes while
		leaving their geometric mean unchanged.	Equation \ref{eq:FlatnessManipulation} (repeated here) gives the new
		spectral flatness for a given gain parameter, $m$.

		\[ \mathrm{SF} = \frac{N\sqrt[N]{\prod_{n \in P} a_{n}^{2}}}
					   {\frac{A_{L}}{m^{2}} + m^{2}A_{H} + A_{E}} \]

		Only the denominator of this equation are changed by $m$. Increasing the denominator will decrease the
		spectral flatness and decreasing it will have to opposite effect. We can simplify this by considering the
		sum of the amplitudes of the two scaled groups of spectral components:

		\[ \frac{A_{L}}{m^{2}} + m^{2}A_{H} \]

		For the spectral flatness to increase this expression should be less than the unscaled amplitudes:

		\begin{gather}
			\frac{A_{L}}{m^{2}} + m^{2}A_{H} < A_{L} + A_{H} \nonumber \\
			\nonumber \\
			A_{L} + m^{4}A_{H} < (A_{L} + A_{H})m^{2} \nonumber \\
			\nonumber \\
			m^{4}A_{H} - (A_{L} + A_{H})m^{2} + A_{L} < 0 \nonumber \\
			\nonumber \\
			(m^{2}A_{H} - A_{L})(m^{2} - 1) < 0 \nonumber \\
			\nonumber \\
			\frac{A_{L}}{A_{H}} < m^{2} < 1 \nonumber
		\end{gather}

		The spectral flatness will increase for values of $m$ in the interval:

		\[ \sqrt{\frac{A_{L}}{A_{H}}} < m < 1 \]

		The maximum spectral flatness occurs at the minimum arithmetic mean:

		\begin{gather}
			y = \frac{A_{L}}{m^{2}} + m^{2}A_{H} \nonumber \\
			\nonumber \\
			\frac{dy}{dx} = -2\frac{A_{L}}{m^{3}} + 2A_{H}m \nonumber \\
			\nonumber \\
			2A_{H}m^{4} = 2A_{L} \nonumber \\
			\nonumber \\
			m^{4} = \frac{A_{L}}{A_{H}} \nonumber \\
			\nonumber \\
			m = \sqrt[4]{\frac{A_{L}}{A_{H}}} \nonumber
		\end{gather}

\chapter{Software Resources}

\end{appendices}
