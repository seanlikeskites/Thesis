\begin{appendices}
\chapter{Mathematical Derivations}
\label{app:MathematicalDerivations}
	\section{Fourier Series of a Full Wave Rectified Sinusoid}
	\label{app:MathematicalDerivations-Rectification}
		The complex Fourier series coefficients, $c_{n}$ of a full wave rectified sinusoid can be calculated as
		follows:

		\[ c_{n} = \frac{1}{2\pi} \int_{-\pi}^{\pi} |\sin(x)|e^{-inx} dx \]

		Expanding out the complex exponential gives:

		\[ c_{n} = \frac{1}{2\pi} \int_{-\pi}^{\pi} \bigl( |\sin(x)|\cos(nx) - i|\sin(x)|\sin(nx) \bigr) dx \]

		The expression $|\sin(x)|\sin(nx)$ is an odd function, so across the interval of our Fourier series integral
		we have:

		\[ \int_{-\pi}^{\pi} i|\sin(x)|\sin(nx) = 0 \]

		Removing this term from the integral leaves us with:

		\[ c_{n} = \frac{1}{2\pi} \int_{-\pi}^{\pi} |\sin(x)|\cos(nx) dx \]

		The expression $|\sin(x)|\cos(nx)$ is an even function, so we can adjust the interval over which we take the
		integral:

		\begin{align}
			c_{n} & = \frac{1}{\pi} \int_{0}^{\pi} \sin(x)\cos(nx) dx \nonumber \\
			\nonumber \\
			& = \frac{1}{2\pi} \int_{0}^{\pi} \bigl( \sin((n+1)x) + \sin((1-n)x) \bigr) dx \nonumber \\
			\nonumber \\
			& = \frac{1}{2\pi} \left[ -\frac{\cos((n+1)x)}{n+1} - 
				\frac{\cos((1-n)x)}{1-n} \right]_{0}^{\pi} \nonumber
		\end{align}

		When $n$ is even:

		\begin{align}
			c_{n} & = \frac{1}{2\pi} \left( \frac{1}{n+1} + \frac{1}{1-n} 
				+ \frac{1}{n+1} + \frac{1}{1-n} \right) \nonumber \\
			\nonumber \\
			& = \frac{1}{2\pi} \frac{2(1-n) + 2(n+1)}{1 - n^{2}} \nonumber \\
			\nonumber \\
			& = \frac{2}{\pi(1 - n^{2})} \nonumber
		\end{align}

		When $n$ is odd:

		\begin{align}
			c_{n} & = \frac{1}{2\pi} \left( -\frac{1}{n+1} - 
				\frac{1}{1-n} + \frac{1}{n+1} + \frac{1}{1-n} \right) \nonumber \\
			\nonumber \\
			& = 0 \nonumber
		\end{align}

		Giving:

		\[ c_{n} = \begin{cases}
				\frac{2}{\pi(1 - n^{2})} & \text{when $n$ is even} \\
				0 & \text{when $n$ is odd}
			\end{cases} \]

	\section{Fourier Series of a Sinusoid after Application of an Integrator}
	\label{app:MathematicalDerivations-Integrator}
		Applying the integrator in Equation~\ref{eq:Integrator} to a sinusoid $\sin(x)$, sampled at a rate of
		$f_{s}$, produces a periodic waveform, $y$, with a period of $2\pi$, defined on the interval $0$ to $2\pi$
		as:

		\[ y = \begin{cases}
				kf_{s} (1 - \cos(x)) & \text{when $0 \leq x < \pi$} \\
				kf_{s} (3 + \cos(x)) & \text{when $\pi \leq x < 2\pi$} \\
			\end{cases} \]

		The complex Fourier series coefficients, $c_{n}$, of this waveform can be calculated as:

		\[ c_{n} = \frac{kf_{s}}{2\pi} \left( \int_{0}^{\pi} (1-\cos(x))e^{-inx} dx
						      + \int_{\pi}^{2\pi} (3 + \cos(x))e^{-inx} dx \right) \]

		Evaluating the first integral gives:
		
		\begin{align}
			\int_{0}^{\pi} (1-\cos(x))e^{-inx} dx & = 
				\int_{0}^{\pi} \left( e^{-inx} - \cos(x)e^{-inx} \right) dx \nonumber \\
			\nonumber \\
			& = \left[ \frac{ie^{-inx}}{n} - \frac{e^{-inx}(\sin(x) 
				- in\cos(x))}{1 - n^{2}} \right]_{0}^{\pi} \nonumber \\
			\nonumber \\
			& = \frac{ie^{-in\pi}}{n} - \frac{ine^{-in\pi}}{1 - n^{2}} 
				- \frac{i}{n} - \frac{in}{1 - n^{2}} \nonumber \\
			\nonumber \\
			& = \frac{i}{n(1-n^{2})} \left( (1 - n^{2})e^{-in\pi} - 
				n^{2}e^{-in\pi} - (1 - n^{2}) - n^2 \right) \nonumber \\
			\nonumber \\
			& = \frac{i}{n - n^{3}} \left( e^{-in\pi} - 2n^{2}e^{-in\pi} - 1 \right) \nonumber
		\end{align}

		And the second integral yields:

		\begin{align}
			\int_{\pi}^{2\pi} (3+\cos(x))e^{-inx} dx & = 
				\int_{\pi}^{2\pi} \left( 3e^{-inx} + \cos(x)e^{-inx} \right) dx \nonumber \\
			\nonumber \\
			& = \left[ \frac{3ie^{-inx}}{n} + \frac{e^{-inx}(\sin(x) - 
				in\cos(x))}{1 - n^{2}} \right]_{\pi}^{2\pi} \nonumber \\
			\nonumber \\
			& = \frac{3i}{n} - \frac{in}{1 - n^{2}} - \frac{3ie^{-in\pi}}{n} - 
				\frac{ine^{-in\pi}}{1 - n^{2}} \nonumber \\
			\nonumber \\
			& = \frac{i}{n(1-n^{2})} \left( 3(1-n^{2}) - n^{2} - 
				3(1-n^{2})e^{-in\pi} - n^{2}e^{-in\pi} \right) \nonumber \\
			\nonumber \\
			& = \frac{i}{n - n^{3}} \left( 2n^{2}e^{-in\pi} - 3e^{-in\pi} - 4n^{2} + 3 \right) \nonumber
		\end{align}

		Combining these results gives:

		\begin{align}
			c_{n} & = \frac{kf_{s}}{2\pi} \times \frac{i}{n - n^{3}} \left( 2 - 2e^{-in\pi} - 4n^{2} \right) 
				\nonumber \\
			\nonumber \\
			& = \frac{ikf_{s}}{\pi} \left( \frac{2n^{2} + e^{-in\pi} - 1}{n^{3} - n} \right) \nonumber
		\end{align}

		When $n = 1$ this will involve a division by 0, so we can find $c_{1}$ as:

		\[ c_{1} = \frac{kf_{s}}{2\pi} \left( \int_{0}^{\pi} (1-\cos(x))e^{-ix} dx
						      + \int_{\pi}^{2\pi} (3 + \cos(x))e^{-ix} dx 
						      \right) \]

		Starting with the first integral:
		
		\begin{align}
			\int_{0}^{\pi} (1-\cos(x))e^{-ix} dx & = \int_{0}^{\pi} \left( e^{-ix} - 
				\cos^{2}(x) + i\cos(x)\sin(x) \right) dx \nonumber \\
			\nonumber \\
			& = \int_{0}^{\pi} \left( e^{-ix} - \frac{1}{2} - 
				\frac{\cos(2x)}{2} + \frac{i\sin(2x)}{2} \right) dx \nonumber \\
			\nonumber \\
			& = \left[ ie^{-ix} - \frac{x}{2} - \frac{\sin(2x)}{4} - 
				\frac{i\cos(2x)}{4} \right]_{0}^{\pi} \nonumber \\
			\nonumber \\
			& = -i - \frac{\pi}{2} - \frac{i}{4} - \left( i - \frac{i}{4} \right) \nonumber \\
			\nonumber \\
			& = -2i - \frac{\pi}{2} \nonumber
		\end{align}

		And the second:
		
		\begin{align}
			\int_{\pi}^{2\pi} (3+\cos(x))e^{-ix} dx & = 
				\int_{\pi}^{2\pi} \left( 3e^{-ix} + \cos^{2}(x) - 
				i\cos(x)\sin(x) \right) dx \nonumber \\
			\nonumber \\
			& = \int_{\pi}^{2\pi} \left( 3e^{-ix} + \frac{1}{2} + \frac{\cos(2x)}{2} - 
				\frac{i\sin(2x)}{2} \right) dx \nonumber \\
			\nonumber \\
			& = \left[ 3ie^{-ix} + \frac{x}{2} + \frac{\sin(2x)}{4} + 
				\frac{i\cos(2x)}{4} \right]_{\pi}^{2\pi} \nonumber \\
			\nonumber \\
			& = 3i + \pi + \frac{i}{4} - \left( -3i + \frac{\pi}{2} + \frac{i}{4} \right) \nonumber \\
			\nonumber \\
			& = 6i + \frac{\pi}{2} \nonumber
		\end{align}

		Which combine to give:

		\[ c_{1} = \frac{2ikf_{s}}{\pi} \]

		Which also allows us to write:

		\[ c_{-1} = -\frac{2ikf_{s}}{\pi} \]

		Finally, when $n = 0$ division by 0 will also occur. $c_{0}$ is calculated as follows:

		\[ c_{0} = \frac{kf_{s}}{2\pi} \left( \int_{0}^{\pi} 1-\cos(x) dx
						      + \int_{\pi}^{2\pi} 3 + \cos(x) dx 
						      \right) \]

		Starting with the first integral:

		\begin{align}
			\int_{0}^{\pi} \bigl( 1 - \cos(x) \bigr) dx & = 
				\biggl[ x + \sin(x) \biggr]_{0}^{\pi} \nonumber \\
			\nonumber \\
			& = \pi \nonumber
		\end{align}

		And the second:

		\begin{align}
			\int_{\pi}^{2\pi} \bigl( 3 + \cos(x) \bigr) dx & = 
				\biggl[ 3x - \sin(x) \biggr]_{\pi}^{2\pi} \nonumber \\
			\nonumber \\
			& = 6\pi - 3\pi \nonumber \\
			\nonumber \\
			& = 3\pi \nonumber
		\end{align}

		Leading to:

		\[ c_{0} = 2kf_{s} \]

		Which gives the Fourier series coefficients as:

		\begin{gather}
			c_{-1} = - \frac{2ikf_{s}}{\pi} \nonumber \\
			c_{0} = 2kf_{s} \nonumber \\
			c_{1} = \frac{2ikf_{s}}{\pi} \nonumber \\
			c_{n} = \frac{ikf_{s}}{\pi} \left( \frac{2n^{2} + e^{-in\pi} - 1}{n^{3} - n} \right) \nonumber
		\end{gather}

	\section{Parameterisation of Spectral Flatness}
	\label{app:MathematicalDerivations-SpectralFlatness}
		Referring back to Section~\ref{sec:FeatureControl-Parameterisation-Flatness}, we recall that the spectral
		flatness was manipulated by altering the arithmetic mean of the spectral components' amplitudes, while
		leaving their geometric mean unchanged.	Equation \ref{eq:FlatnessManipulation} (repeated here) gives the new
		spectral flatness for a given gain parameter, $m$.

		\[ \mathrm{SF} = \frac{N\sqrt[N]{\prod_{n \in P} a_{n}^{2}}}
					   {\frac{A_{L}}{m^{2}} + m^{2}A_{H} + A_{E}} \]

		Only the denominator of this equation is changed by $m$. Increasing the denominator will decrease the
		spectral flatness, and decreasing it will have to opposite effect. We can simplify this by considering the
		sum of the amplitudes of the two scaled groups of spectral components:

		\[ \frac{A_{L}}{m^{2}} + m^{2}A_{H} \]

		For the spectral flatness to increase, this expression should be less than the unscaled amplitudes:

		\begin{gather}
			\frac{A_{L}}{m^{2}} + m^{2}A_{H} < A_{L} + A_{H} \nonumber \\
			\nonumber \\
			A_{L} + m^{4}A_{H} < (A_{L} + A_{H})m^{2} \nonumber \\
			\nonumber \\
			m^{4}A_{H} - (A_{L} + A_{H})m^{2} + A_{L} < 0 \nonumber \\
			\nonumber \\
			(m^{2}A_{H} - A_{L})(m^{2} - 1) < 0 \nonumber \\
			\nonumber \\
			\frac{A_{L}}{A_{H}} < m^{2} < 1 \nonumber
		\end{gather}

		The spectral flatness will increase for values of $m$ in the interval:

		\[ \sqrt{\frac{A_{L}}{A_{H}}} < m < 1 \]

		The maximum spectral flatness occurs at the minimum arithmetic mean:

		\begin{gather}
			y = \frac{A_{L}}{m^{2}} + m^{2}A_{H} \nonumber \\
			\nonumber \\
			\frac{dy}{dx} = -2\frac{A_{L}}{m^{3}} + 2A_{H}m \nonumber \\
			\nonumber \\
			2A_{H}m^{4} = 2A_{L} \nonumber \\
			\nonumber \\
			m^{4} = \frac{A_{L}}{A_{H}} \nonumber \\
			\nonumber \\
			m = \sqrt[4]{\frac{A_{L}}{A_{H}}} \nonumber
		\end{gather}

\chapter{Software Resources}
	As part of the work undertaken for this thesis several pieces of software were developed. This appendix lists these,
	and provides links to where they can be accessed, where available.
	
	\section{The \acrshort{safe} Project Codebase}
		The largest software element developed for this thesis was the suite of \acrshort{safe} plug-ins, developed
		in C++ using the JUCE framework. This code implements audio processing algorithms, as well as the collection
		of semantic information (audio features and metadata) and the serialisation of that data as XML. Source code
		for the plug-ins can be found on GitHub
		(\href{https://github.com/semanticaudio/SAFE}{https://github.com/semanticaudio/SAFE}). The underlying code
		of these was also packaged into a module for the JUCE framework, easing the development of new plug-ins with
		the \acrshort{safe} functionality
		(\href{https://github.com/seanlikeskites/SAFEJuceModule}{https://github.com/seanlikeskites/SAFEJuceModule}).

		The \acrshort{safe} plug-ins send all the semantic information they collect to a web server, via HTTP
		requests. The back end code to handle these requests was written in PHP and parses the plug-in data for
		storage in a MySQL database.

		For the analysis of the \acrshort{safe} data, a package for the statistical programming language R was
		developed. This package interfaces with the \acrshort{safe} database and provides functions for some of the
		analysis carried out in Chapter~\ref{chap:TimbreEvaluation}. This code is not publicly available, as it
		contains details which would compromise the security of the \acrshort{safe} database. R code for a large
		proportion of the analysis carried out in Chapter~\ref{chap:TimbreEvaluation} can be found on GitHub
		(\href{https://github.com/seanlikeskites/Thesis/tree/master/RPlots/SAFEAnalysis}
		{https://github.com/seanlikeskites/Thesis/tree/master/RPlots/SAFEAnalysis}).

	\section{Harmonic Excitation Evaluation and Development Codebase}
		The evaluation and development of harmonic excitation systems, in Chapters~\ref{chap:ExcitationEvaluation}
		and \ref{chap:FeatureControl}, was all undertaken in GNU Octave. Each of the harmonic excitation methods
		discussed was implemented, along with feature extraction code, to examine their effects on low level audio
		features. The source code for this can be found on Bitbucket
		(\href{https://bitbucket.org/seanlikeskites/HarmonicExcitationOctave}
		{https://bitbucket.org/seanlikeskites/HarmonicExcitationOctave}). The plots for this chapter were also
		initially produced using GNU Octave, but eventually the data was imported into R to make the plotting
		consistent throughout the thesis.

	\section{Perceptual Listening Test Codebase}
		Additional software was also developed for the perceptual listening tests performed in
		Chapter~\ref{chap:PerceptualExperiments}. The \acrshort{mushra} test, discussed in
		Section~\ref{sec:PerceptualExperiments-Reconstruction}, was undertaken using MATLAB. This involved the
		development of a framework for performing \acrshort{mushra} tests in MATLAB, source code for which is
		available on SourceForge
		(\href{https://sourceforge.net/projects/matlabmushra}{https://sourceforge.net/projects/matlabmushra}).

		The semantically controlled plug-ins developed, and evaluated, in
		Section~\ref{sec:PerceptualExperiments-SemanticControl} were developed in C++ using the JUCE framework.
		These plug-ins reused the \acrshort{safe} codebase, allowing for them to perform the feature analysis used
		to evaluate their performance.

\chapter{Harmonic Excitation System Design Notes}
\label{app:FilterNotes}
	In Section~\ref{sec:FeatureControl-Systems}, several harmonic excitation systems were proposed which provide
	intuitive control over the spectral structure of a signal. This appendix provides further detail about the use of
	these systems, specifically the filtering used within them.

	\section{$f_{0}$ Isolation Filters}
		As illustrated in Figure~\ref{fig:F0Tracking}, the proposed harmonic excitation systems all rely on the
		isolation of a signal's $f_{0}$ using a band pass filter. Throughout this thesis, this is implemented using
		a second order \acrshort{iir} (biquad) band pass filter, centered at the frequency of the input's $f_{0}$.
		This has been shown to be sufficient for excitation of low order harmonics, in signals with little
		inharmonic energy.  However, as shown in Figure~\ref{fig:CelloFilterOrderSpectra}, usage of a higher order
		\acrshort{fir} filter reduces the unwanted intermodulation components significantly. This \acrshort{fir}
		filter was implemented as a windowed sinc filter, using a Blackman window.

	\section{High Order Harmonic Filtering}
		The purpose of the filter after the nonlinear device in Figure~\ref{fig:SpectralShapingSystem} is to
		isolate, and shape, the higher order excited harmonics. Principally, this filter should have a high pass
		element, attenuating frequencies below the tenth harmonic. This avoids superposition issues with the first
		nine individually generated harmonics. For use in this thesis, this is implemented using a biquad high pass
		filter, with the cutoff frequency set at $9.5f_{0}$, which proved sufficient. Additional filtering may be
		applied to creatively shape these high order harmonics; for instance, one might apply a low pass filter in
		order to change the spectral slope of the signal.

		The low pass filter added before the nonlinear device in Figure~\ref{fig:InharmonicitySystem} is intended to
		allow the high order harmonics to contain inharmonic partials. This can be implemented as a biquad low pass,
		the cutoff frequency of which will determine the inharmonicity present in the output of the nonlinear
		device. A cutoff at the $f_{0}$ will produce the minimum inharmonicity, while higher cutoff frequencies will
		increase the inharmonicity.

	\section{Superposition Filtering}
		The filtering applied to the original signal in Figure~\ref{fig:SuperpositionSystem} is intended to remove
		energy at any of the harmonics which are being excited. This is achieved through series biquad filters, each
		configured as a notch filter centered at the relevant harmonic's frequency. If high order harmonics are also
		to be excited, a low pass filter can be included, removing high order harmonic content from the input
		signal.

\end{appendices}
