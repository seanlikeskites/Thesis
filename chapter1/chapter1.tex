%\setcounter{chapter}{0}
\chapter{Introduction}
\label{chap:Introduction}
	\note
	{
		In music production people use audio effects to manipulate the timbre of pieces of audio. One such effect is
		harmonic excitation. Distortion is typically applied to guitar signals.
	}

\section{Motivation}
\label{sec:Introduction-Motivation}
% Genuine Waffle (if those who disagree have uncovered this comment I have no regrets)
	Traditional audio effects evolved from the use of analogue signal processing systems designed to address the
	technical requirements of broadcast or recording. As such, these effects have control parameters which relate
	directly to specific technical properties of an audio signal: a traditional equaliser has parameters for selecting
	specific frequency regions, traditional dynamics processors (compressors and limiters) have parameters relating to
	signal amplitudes. These parameters are effective when correcting technical issues with a signal but do not allow
	for intuitive creative manipulation of timbre.

	The application of audio effects for creative purposes is typically motivated by the perceived timbre of the audio.
	Creative decisions are often expressed using language which does not refer to the mathematical properties of the
	recorded signals. For example one might ask that the saxophone is made more present in the mix, or that the piano be
	made to sound more airy. In order to complete these tasks using traditional audio effects music, producers and audio
	engineers require knowledge of how the technical parameters relate to specific perceived aspects of timbre. This
	knowledge is gained through training and experience, presenting an obstacle to novices.

	\note
	{
		Presets are there to help novice users but they are shit. One must still know what kind of processing they
		wish to apply. They are also highly signal dependant.

		Proper intuitive systems are based on the relationship between timbre and mathematical properties.
		These have been proposed for different types of effect but we will focus on distortion.
	}

\section{Objectives}
\label{sec:Introduction-Objectives}
	% Write this properly.
	The underlying objective of this thesis is to produce intuitive timbre shaping effects based on harmonic excitation
	algorithms. This is achieved through answering the following questions:

	\begin{itemize}
		\item What semantic terms are commonly used to describe the timbre of harmonic excitation effects? 
		\item What features of audio signals contribute to these timbral descriptions? Investigation of what
		      alterations are made to signals to produce a certain timbral result.
		\item How can harmonic excitation systems be used to control these features? Systems will be proposed which
		      allow control over particular semantic features and the accuracy of these systems assessed.
	\end{itemize}

\section{Thesis Structure}
\label{sec:Introduction-ThesisStructure}
	In {\bf{Chapter~\ref{chap:Timbre}}} the background of research into timbre is discussed. It begins with a discussion
	of the low level features of audio signals. Popular experimental methodologies for uncovering the perceptual effects
	of these features are then introduced along with the methods used to analyse their results. Finally, existing work
	concerning the control of perceptual features of audio signals is discussed.

	{\bf{Chapter~\ref{chap:Excitation}}} covers the body of work concerning harmonic excitation, starting with a
	discussion of the analysis of nonlinear systems in the field of audio production. The uses of such systems and the
	their timbral effects are then presented. Finally, various algorithms for harmonic excitation, proposed in existing
	literature, are described.

	In {\bf{Chapter~\ref{chap:TimbreEvaluation}}} the collection of semantic audio data during the production process is
	discussed. This data is used to determine the language used for describing audio in a studio environment and how
	audio effects are used to produce certain timbral effects.

	In {\bf{Chapter~\ref{chap:ExcitationEvaluation}}} a set of criteria for assessing the applicability of harmonic
	excitation techniques to timbral control is proposed. The algorithms described in the Chapter~\ref{chap:Excitation}
	are then evaluated against these criteria to identify those most suitable for use in timbral control systems.
	Methods by which the performance of certain harmonic excitation methods can be improved are proposed.

	In {\bf{Chapter~\ref{chap:FeatureControl}}} results from Chapters \ref{chap:TimbreEvaluation} and
	\ref{chap:ExcitationEvaluation} are used to inform the design of systems for controlling low level features using
	harmonic excitation techniques.

	In {\bf{Chapter~\ref{chap:PerceptualExperiments}}} perceptual effects of the systems proposed in
	Chapter~\ref{chap:FeatureControl} are evaluated. 

\section{Contributions}
\label{sec:Introduction-Contributions}

	\begin{itemize}
		\item A new method for the collection of semantic audio descriptors and their underlying features
		      (Chapter~\ref{chap:TimbreEvaluation}).
		\item A methodology for the assessment of harmonic generation algorithms for use in timbral control
		      (Chapter~\ref{chap:ExcitationEvaluation}).
		\item A comparison of harmonic generation algorithms and their suitability to provide perceptual control
		      (Chapter~\ref{chap:ExcitationEvaluation}).
		\item The development of new harmonic excitation systems (Chapter~\ref{chap:FeatureControl}).
		\item Potential new methods for controlling perceptual attributes using harmonic excitation
		      (Chapter~\ref{chap:FeatureControl}).
	\end{itemize}

	Associated Publications:

	\begin{itemize}
		\item \bibentry{enderby2012harmonic}.
		\item \bibentry{enderby2013methods}.
		\item \bibentry{stables2014safe}.
	\end{itemize}
