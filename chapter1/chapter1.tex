%\setcounter{chapter}{0}
\chapter{Introduction}
\label{chap:Introduction}

\section{Motivation}
\label{sec:Introduction-Motivation}
% Genuine Waffle (if those who disagree have uncovered this comment I have no regrets)
	\note
	{
		Talk about the training needed to use studio equipment. Musicians need technical assistance to create the
		timbres they desire. Can we assist musicians by making studio equipment more intuitive.
	}

\section{Objectives}
\label{sec:Introduction-Objectives}
	% Write this properly.
	The underlying objective of this thesis is to produce intuitive timbre shaping effects based on harmonic excitation
	algorithms. This is achieved through answering the following questions:

	\begin{itemize}
		\item What semantic terms are commonly used to describe the timbre of harmonic excitation effects? 
		\item What features of audio signals and contribute to these timbral descriptions? Investigation of what
		      alterations are made to signals to produce a certain timbral result.
		\item How can harmonic excitation systems be used to control these features? Systems will be proposed which
		      allow control over particular semantic features and the accuracy of these systems assessed.
	\end{itemize}

\section{Thesis Structure}
\label{sec:Introduction-ThesisStructure}
	In {\bf{Chapter \ref{chap:Timbre}}} the background of research into timbre is discussed. It begins with a discussion
	of the low level features of audio signals. Popular experimental methodologies for uncovering the perceptual effects
	of these features are then introduced along with the methods used to analyse their results. Finally, existing work
	concerning the control of perceptual features of audio signals is discussed.

	{\bf{Chapter \ref{chap:Excitation}}} covers the body of work concerning harmonic excitation. Starting with a
	discussion of the analysis of nonlinear systems in the field of audio production. The uses of such systems and the
	their timbral effects are then presented. Finally, various algorithms for harmonic excitation, proposed in existing
	literature, are described.

	{\bf{Chapter \ref{chap:TimbreEvaluation}}} the collection of semantic audio data during the production process is
	discussed. Data collected from music producers is used to determine the language used to describe audio in a studio
	environment and how audio effects are used to produce certain timbral effects.

	In {\bf{Chapter \ref{chap:ExcitationEvaluation}}} a set of criteria for assessing the applicability of
	harmonic excitation techniques to timbral control is proposed. The algorithms described in the previous chapter are
	then evaluated against these criteria to identify those most suitable for use in subsequent chapters. Methods by
	which the performance of certain harmonic excitation methods can be improved are proposed.

	In {\bf{Chapter \ref{chap:FeatureControl}}} results from Chapter \ref{chap:TimbreEvaluation} are used to inform
	the design of systems for controlling low level features using harmonic excitation techniques.

	In {\bf{Chapter \ref{chap:PerceptualExperiments}}} perceptual effects of the systems proposed in Chapter
	\ref{chap:FeatureControl} are evaluated. 

\section{Contributions}
\label{sec:Introduction-Contributions}

	\note{Still unsure on the order of these.}

	\begin{itemize}
		\item A new method for the collection of semantic audio descriptors and their underlying features (Chapter
		      \ref{chap:TimbreEvaluation}).
		\item A methodology for the assessment of harmonic generation algorithms (Chapter
		      \ref{chap:ExcitationEvaluation}).
		\item A comparison of harmonic generation algorithms and their suitability to provide perceptual control
		      (Chapter \ref{chap:ExcitationEvaluation}).
		\item The development of new harmonic excitation systems (Chapter \ref{chap:FeatureControl}).
		\item Potential new methods for controlling perceptual attributes (Chapter \ref{chap:FeatureControl}).
	\end{itemize}

	Associated Publications:

	\begin{itemize}
		\item \bibentry{enderby2012harmonic}.
		\item \bibentry{enderby2013methods}.
	\end{itemize}

\note{Probably want some section denoting the mathematical notation used throughout the thesis.}
