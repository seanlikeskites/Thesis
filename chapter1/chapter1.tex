%\setcounter{chapter}{0}
\chapter{Introduction}
\label{chap:Introduction}

\section{Motivation}
\label{sec:Introduction-Motivation}
% Genuine Waffle (if those who disagree have uncovered this comment I have no regrets)
	\note
	{
		Talk about the training needed to use studio equipment. Musicians need technical assistance to create the
		timbres they desire. Can we assist musicians by making studio equipment more intuitive.
	}

\section{Objectives}
\label{sec:Introduction-Objectives}
	\begin{itemize}
		\item What semantic terms are often used to describe the timbre of harmonic excitation effects?
		\item What features of the audio signals and processing algorithms contribute to these timbral descriptions?
		\item How can harmonic excitation systems be used to control these features.
	\end{itemize}

\section{Thesis Structure}
\label{sec:Introduction-ThesisStructure}
	In {\bf{Chapter \ref{chap:Timbre}}} the background of research into timbre is discussed. It begins with a discussion
	of the low level features of audio signals. Popular experimental methodologies for uncovering the perceptual effects
	of these features are then introduced along with the methods used to analyse their results. Finally, existing work
	concerning the control of perceptual features of audio signals is discussed.

	{\bf{Chapter \ref{chap:Excitation}}} covers the body of work concerning harmonic excitation. Starting with a
	discussion of the analysis of nonlinear systems in the field of audio production. The uses of such systems and the
	their timbral effects are then presented. Finally, various algorithms for harmonic excitation, proposed in existing
	literature, are described.

	In {\bf{Chapter \note{(doesn't exist separately yet)}}} a set of criteria for assessing the applicability of
	harmonic excitation techniques to timbral control is proposed. The algorithms described in the previous chapter are
	then evaluated against these criteria to identify those most suitable for use in subsequent chapters. Methods by
	which the performance of certain harmonic excitation methods can be improved are proposed.

	\note{Perhaps merge that and the following chapter.}

	In {\bf{Chapter \ref{chap:FeatureControl}}} systems comprised of the algorithms discussed in the previous
	chapter are proposed. Methods by which these systems can be used to manipulate various low level audio features are
	then discussed.

	{\bf{Chapter \ref{chap:ListeningTests}}} the collection of semantic audio data during the production process is
	discussed. Data collected from music producers is used to determine the language used to describe audio in a studio
	environment and how audio effects are used to produce certain timbral effects.

	In {\bf{Chapter \ref{chap:PerceptualControl}}} results from Chapter \ref{chap:ListeningTests} are used to inform the
	design of systems for controlling perceptual features using harmonic excitation techniques. The success of the
	resulting systems is evaluated using a set of subjective listening tests.

\section{Associated Publications}
\label{sec:Introduction-AssociatedPublications}
