%Suitability of Exciters for Perceptual Control
%	Good features of a timbral control algorithm.
%		Consistent effect across amplitudes (DAFx paper)
%		Flexibility of single harmonic control (SMC paper)
%	Parameterise feature changes in terms of harmonic excitation.
%	Most suitable methods for given feature manipulations.
%	Easiest features to control in isolation.

\chapter{Suitability of Exciters for Perceptual Control}
\label{chap:FeatureControl}

	In order to aid in the intuitiveness of an audio effect it should produce a similar perceptual effect across a wide range of input signals. This is not the case with traditional audio signal processing methods. Take the EQ for example. We can set up an EQ to boost some frequencies in a desired range. A problem arises where we process signals which have no energy in this frequency range. For some signals there will be a noticeable change in the spectral characteristics, but other signals will remain unchanged.

	This problem is compounded when the audio processing in nonlinear. As well as different signals producing different perceptual effect, now the same signal at different levels might produce different perceptual effects. A simple example of this is a peak clipping effect. If a signal's amplitude is above the clipping threshold there will be a noticeable effect. This signal will be unaffected however, if its amplitude falls below the threshold.

	\note{Mention of homogeneous nonlinear systems \citep{larsen2004audio}.}

	The field of adaptive audio effects looks to solve some of these issues by producing effects which analyses the input signal and adjust their parameters accordingly \note{[cite some stuff up]}.

	As discussed in Chapter \ref{chap:Timbre} previous studies have attempted to control perceptual features by controlling specific audio features. This chapter will identify the audio features which can be controlled through harmonic excitation and which excitation algorithms are the most suitable for given feature changes.

\section{Evaluating Excitation Methods}
\label{sec:FeatureControl-MethodEvaluation}

	\note{Consistent effects across amplitudes a la DAFx paper.
	      Flexibility introduced by allowing single harmonic control a la SMC paper}

	Several harmonic excitation methods were discussed in Section \ref{sec:Excitation-Methods}. When applied to the task of controlling specific audio features each of these methods has its own advantages and disadvantages.

	\subsection{Invariance to Amplitude}
	\label{sec:FeatureControl-AmplitudeInvariance}

		\subsubsection*{Static Nonlinearities}		
			As previously mentioned simple static nonlinearities are very susceptible to change in input amplitude. \citet{deman2014adaptive} counteract this issue by having the clipping threshold adapt to changes in the RMS amplitude of the input. The user is then provided with a `relative threshold' parameter on which the same setting should give similar perceptual results no matter what the input amplitude.

			\note{Talk about the issues with static nonlinearities raised in \citet{enderby2012harmonic}}

		\subsubsection*{Bandwidth Extension}
			These methods are all homogeneous.
			
		\subsubsection*{Single Harmonic Generation}
			Using single sideband automodulation the proportion of new frequency components in the output signal increases as the input amplitude increases. The instantaneous amplitude and phase and phase vocoder techniques are more robust in this respect.

	\subsection{Flexibility}
	\label{sec:FeatureControl-Flexibility}

		\note{Flexibility is provided by individual harmonic generation \citep{enderby2013methods}}	

	\subsection{Complexity}
	\label{sec:FeatureControl-Complexity}
		
		\note{It is advantageous to use an algorithm that will create accurate harmonics with little analysis.
		      Most methods don't rely on knowing the fundamental in order to generate harmonics. Spectral
		      shifting on the other hand does.}

		\note{Most of the algorithms can be easily improved through the use of a low pass filter, increasing
	             complexity.}

\section{Parameterisation of Feature Changes}
\label{sec:FeatureControl-Parameterisation}
	\todo{This needs doing}

\section{Controlling features with Exciters}
\label{sec:FeatureControl-Control}
	\todo{Should fall out of the previous section}
