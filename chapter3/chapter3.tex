\chapter{Harmonic Excitation}
\label{chap:Excitation}

\section{Introduction}
\label{sec:Excitation-Introduction}
	Nonlinear distortion is an inherent part of an analogue signal path. It alters the dynamic variation of signals and
	introduces new spectral components. Its use as a creative effect is best known by guitar players
	\citep{dutilleux2011nonlinear}. This was due to the use of values in early guitar amplifiers imparting nonlinear
	transforms onto the audio signal. This distorted sound became desirable and several electronic circuits were
	developed with the purpose of inducing nonlinear distortion. More recently research into modelling these distortion
	circuits in the digital domain has been carried out, such as that done by \citet{pakarinen2009a}. Several
	researchers have also worked on creating novel digital distortion techniques \citep{fernandez-cid2001distortion,
	pekonen2008coefficient, puckette2007patch}.

	Distortion is a very broad term which is often used to describe unwanted effects. For this work a better
	defined term, harmonic excitation, will be used to refer to the deliberate and controlled application of nonlinear
	systems in order to introduce new frequency components to a signal. \citet{dutilleux2011nonlinear} defines
	excitation as the process of controlling timbre through the emphasis of certain frequencies. While this is possible
	with linear systems, such as equalisers, nonlinear systems provide more flexibility as they can add energy in areas
	of the spectrum where the original signal had none.

%	This chapter will review how nonlinear systems have been analysed in previous audio research, how their effects are
%	quantified and the timbral transformations they impart. The current uses of harmonic excitation in audio
%	engineering are then discussed in Section \ref{sec:ExcitationEvaluation-Uses}. A list of criteria with which to
%	evaluate harmonic excitation methods for use in real time timbral control is developed in Section
%	\ref{sec:ExcitationEvaluation-Evaluation}.  Section \ref{sec:ExcitationEvaluation-Methods} then describes several
%	methods for harmonic excitation and how they perform in these areas.  Issues with certain methods are identified
%	and ways to overcome them suggested.

\section{Analysis of Nonlinear Systems}
\label{sec:Excitation-AnalysisOfNonlinearSystems}
	A system is nonlinear if it does not satisfy the conditions met by the superposition principle. These conditions
	are that of additivity (Equation \ref{eq:additivity}) and homogeneity (Equation \ref{eq:homogeneity}).

	\begin{equation}
		f(x + y) = f(x) + f(y) 
		\label{eq:additivity}
	\end{equation}

	\begin{equation}
		f(ax) = af(x)
		\label{eq:homogeneity}
	\end{equation}

	Where $f$ is the system, $x$ and $y$ two inputs and $a$ some scaling factor.

	The behaviour of a system which meets these criteria, and who's output does not depend on time (time invariance),
	can be entirely summarised by its response to an impulse signal \citep{phillips2007signals}. Once the impulse
	response of a linear time invariant (LTI) system has been measured the systems response to any arbitrary input is
	known. 

	Analysis of nonlinear systems is more complicated. Unlike LTI systems the properties of the system cannot be
	summarised by the systems response to a single input signal. Depending on the field in which a nonlinear system is
	being studied different approaches are taken in an attempt to summarise the behaviour of nonlinear systems.

	In the field of audio engineering much of the literature concerning nonlinearities has to do with minimising
	distortion or measuring the maximum allowable levels of distortion. Several distortion metrics have been developed
	each with there own uses. A review of many of these metrics is given by \cite{voishvillo2006assessment}, some of
	which will be discussed in Section \ref{sec:Excitation-Analysis-Metrics}.

	Where more information is needed nonlinear modelling techniques are used. These revolve around setting the
	parameters of a generalised nonlinear system such that its behaviour matches that of the system being analysed. The
	nonlinear modelling techniques utilised in the audio processing literature are discussed in Section
	\ref{sec:Excitation-Analysis-Modelling}.

	\subsection{Objective Distortion Metrics}
	\label{sec:Excitation-Analysis-Metrics}
		Total Harmonic Distortion (THD) and Intermodulation Distortion (IMD) are the traditional objective measures
		of distortion \citep{czerwinski2001multitone1}. THD measures the level of new harmonic components
		introduced to a sinusoidal signal. There are two different ways in which THD is calculated. These have been
		denoted THD\sub{F} and THD\sub{R} by \citet{shmilovitz2005on} and are calculated using Equations
		\ref{eq:thdf} and \ref{eq:thdr} respectively.

		\begin{equation}
			\textrm{THD}_{\textrm{F}} = \frac{\sqrt{\sum_{n = 2}^{\infty} A_{n}^{2}}}{A_{1}}
			\label{eq:thdf}
		\end{equation}

		\begin{equation}
			\textrm{THD}_{\textrm{R}} = \sqrt{\frac{\sum_{n = 2}^{\infty} A_{n}^{2}}
			                                       {\sum_{n = 1}^{\infty} A_{n}^{2}}}
			\label{eq:thdr}
		\end{equation}

		Where $A_n$ is the amplitude of the $n$\super{th} harmonic. 

		Each of these measures has slightly different properties. THD\sub{R} measure the proportion of the output
		signal which consists of new harmonic components, producing a value between zero and one. THD\sub{F}
		measures the relative levels of new harmonic content and the energy at the frequency of the original
		sinusoid producing a positive value. \citet{shmilvoltiz2005on} suggests that THD\sub{F} provides a better
		measure for signals with high distortion as there is no upper bound on the value which is produced.

		Both these methods been used in recent work published in the audio field (THD\sub{F} by
		\citet{fleischmann2014a} and THD\sub{R} by \citet{dutilleux2011nonlinear}). The lack of standardisation for
		this metric makes it difficult to compare experimental results reported by different sources.

		IMD is a measure of the new spectral content introduced by a system as a result of intermodulation between
		the frequency components of the original signal. There are several different standards for the calculation
		of IMD some of which are listed by \citet{voishvillo2006assessment}.

		THD and IMD give very limited information about the response of nonlinear systems. They are typically
		measured using simple input signals which are not representative of the signals a system would process in
		actual use. Given that nonlinear systems may not satisfy the condition of superposition a system may
		process simple and complex signals in vastly differing manners. 

		These metrics also give no indication of the perceived degradation in audio quality a system introduces. A
		signal with a high THD values may potentially sound less distorted that one with a small THD. Several
		researchers have developed new distortion metrics which indicate the perceived amount of distortion.

		\citet{geddes2003auditory} suggest some psychoacoustic principles which may be applicable to the analysis
		of nonlinear distortion:

		\begin{itemize}
			\item New frequency components introduced, with lower frequencies than the original components,
				will be more perceptible.
			\item Higher order nonlinearity artefacts will be more perceptible.
			\item Nonlinearities which affect lower amplitude signals will be more perceptible.
		\end{itemize}

		\citet{voishvillo2006assessment} adds that the perception of distortion is decreased at the very high and
		low ends of the frequency spectrum.

		A subset of nonlinear systems can be described by a characteristic curve (a function mapping the
		instantaneous amplitude of the input signal to the input amplitude of the output signal). These systems are
		referred to as static nonlinearities and are memoryless systems (their behaviour is only governed by the
		instantaneous amplitude of the input, not any of its previous states).

		The GedLee metric, proposed by \citet{geddes2003auditory}, measures how much a static nonlinearity will be
		perceived through analysis of its characteristic curve. It is calculated using Equation \ref{eq:gedlee}.

		\begin{equation}
			G_{m} = \sqrt{\int_{-1}^{1} \left( \cos \left( \frac{x\pi}{2} \right) \right)^{2}
				      \left( \frac{d^{2}}{dx^{2}} T(x) \right)^{2} dx}
			\label{eq:gedlee}
		\end{equation}

		Where $T(x)$ is the characteristic curve of the static nonlinearity in question.

		A considerable advantage of the GedLee metric is that it directly measures the system in question rather
		than signals processed by it. This should allow measurements taken using the metric to describe the audible
		degradation of any signal processed by a system. \citet{lee2003auditory} provide results of an experiment
		in which their metric was tested against THD and IMD as a rating of audio quality. They show that for low
		levels of distortion the GedLee metric correlates with subjective audio quality ratings. THD and IMD are
		both found to not correlate well with perceived quality.

		The GedLee metric is only applicable to the analysis of static nonlinearities. Many nonlinear systems are
		more complex than this and may exhibit frequency dependant or time varying behaviour.  While the GedLee
		metric may provide a good, signal independent, measure of distortion, it is only applicable to a subset of
		nonlinear systems.

		\citet{tan2003the} also developed a perceptual metric for distortion, Distortion Score (DS), which they
		then improved upon to create the R\sub{nonlin} metric \citep{tan2004predicting}. Both these metrics rely on
		analysing signals which have been processed, so they may not give results which can be applied as generally
		as those using the GedLee metric. Where these metrics have an advantage is that they use models of the
		human hearing system to better approximate the perceived level of distortion.

		To calculate the DS the audio is split into frequency bands which model the auditory filters of the
		cochlear. These auditory filters represent bands within which two tones will interfere with the perception
		of each other \citep{fastl2007psychoacoustics}. Tones can be masked (made inaudible) by other tones within
		the same auditory filter band. By processing the audio in this manner the DS metric can take account of
		elements of the distortion which may not be perceptible because of masking.
		
		The R\sub{nonlin} metric improves on this model of the hearing system by including a filter with a
		frequency response similar to that of the outer and middle ear. The filter used is as described by
		\citet{glasberg2002a}, it attenuates frequencies at either end of the audible spectrum. This models the
		decreased perception of distortion at these frequencies as reported by \citet{voishvillo2006assessment}.

		\citet{tan2004predicting} report greater correlation between R\sub{nonlin} and perceived audio quality then
		\citet{lee2003auditory} do for the GedLee metric. They also demonstrate how accurate the R\sub{nonlin}
		metric is in predicting the perceived distortion level introduced to music and speech signals.

		The research discussed in this section has focused on measuring the extent to which unwanted distortion can
		be perceived. This does not provide enough information for the description of timbre. The listening tests
		performed to verify the developed metrics asked subjects to rate the quality of audio samples. It is
		possible that nonlinear distortion could alter the timbre of audio without deteriorating its perceived
		quality. Listeners could be asked to describe the timbre of different nonlinear distortions further rather
		than ranking them on the same distortion scale. Previous research has been carried out into the timbre of
		distortion as discussed in Section \ref{sec:Excitation-Timbre}.

	\subsection{Nonlinear Modelling}
	\label{sec:Excitation-Analysis-Modelling}
		Various techniques for modelling nonlinear systems have been developed in the mathematics and engineering
		literature. These models allow for more in depth analysis of a system as well as replicating its effects in
		the digital domain. Summaries of some of these modelling techniques can be found in the work by
		\citet{janczak2005identification} and \citet{ogunfunmi2007adaptive}.

		Two nonlinear modelling techniques have found use in audio processing:

		\begin{itemize}
			\item Wave Digital Filters as discussed by \citet{fettweis1986wave}.
			\item The Synchronised Swept Sine Method as described by \citet{novak2010nonlinear}.
		\end{itemize}

		These techniques are summarised here.

		\subsubsection{Wave Digital Filters}
			Wave digital filters are a class of filter which can be used to emulate analogue circuits. The
			process of creating a digital representation of a circuit involves constructing a `tree' of blocks.
			These blocks represent either electronic components, or connections between components in a
			circuit. Each block obeys a certain set of rules when presented with a signal. Blocks have been
			developed for modelling nonlinear circuit elements such as operational amplifiers and diodes
			\citep{paiva2012emulation}.

			While wave digital filters can accurately represent a nonlinear system there are some
			disadvantages.  As systems get more complex traversing the `tree' of blocks in order to calculate
			the output signal takes more computation. This presents problems when real time system response is
			needed. There are also certain circuit topologies which cannot be represented in the wave digital
			domain \citep{valimaki2011virtual}. Another consideration is that knowledge of the circuitry inside
			a system being modelled is needed. This might not always be available.

		\subsubsection{The Synchronised Swept Sine Method}
			The synchronised swept sine method is a technique for identifying nonlinear systems without prior
			knowledge of their operation. A nonlinear impulse response is generated by analysing the systems
			response to an signal with exponentially increasing frequency as described by
			\citet{novak2010nonlinear}. The result of the analysis is a series of filter kernels to be used in
			the model shown in Figure \ref{fig:hammerstein}.

			\begin{figure}[h!]
				\centering
				\begin{tikzpicture}
					\node (Sig2) [draw] at (1, 3) {$x[n]^{2}$};
					\node (Sig3) [draw] at (1, 2) {$x[n]^{3}$};
					\node (SigN) [draw] at (1, 0.5) {$x[n]^{N}$};

					\node (Filter1) [draw] at (3, 4) {$A_{1}(f)$};
					\node (Filter2) [draw] at (3, 3) {$A_{2}(f)$};
					\node (Filter3) [draw] at (3, 2) {$A_{3}(f)$};
					\node (FilterN) [draw] at (3, 0.5) {$A_{N}(f)$};

					\draw (Sig2) -- (Filter2);
					\draw (Sig3) -- (Filter3);
					\draw (SigN) -- (FilterN);

					\draw [dots] (Sig3) -- (SigN);
					\draw [dots] (Filter3) -- (FilterN);

					\coordinate (Out1) at (4.5, 4);
					\coordinate (Out2) at (4.5, 3);
					\coordinate (Out3) at (4.5, 2);
					\coordinate (OutN) at (4.5, 0.5);

					\draw (Filter1) -- (Out1);
					\draw (Filter2) -- (Out2);
					\draw (Filter3) -- (Out3);
					\draw (FilterN) -- (OutN);

					\node (Add) [operator] at (5, 2.25) {+};
					\draw (Out1) -- (Add);
					\draw (Out2) -- (Add);
					\draw (Out3) -- (Add);
					\draw (OutN) -- (Add);

					\coordinate (In1) at (-0.5, 4);
					\coordinate (In2) at (-0.5, 3);
					\coordinate (In3) at (-0.5, 2);
					\coordinate (InN) at (-0.5, 0.5);

					\draw (In1) -- (Filter1);
					\draw (In2) -- (Sig2);
					\draw (In3) -- (Sig3);
					\draw (InN) -- (SigN);
					\draw (InN) -- (In1);

					\node (In) at (-1.25, 2.25) {$x[n]$};
					\coordinate (InMid) at (-0.5, 2.25);
					\draw (In) -- (InMid);

					\node (Out) at (6, 2.25) {$y[n]$};
					\draw (Add) -- (Out);

				\end{tikzpicture}
				\caption{Generalised Polynomial Hammerstein Model.}
				\label{fig:hammerstein}
			\end{figure}

			The model is, in effect, an extension of a Taylor Series of order $N$. The input is raised to each
			individual power, 1 through $N$, and each of these signals is filtered by an individual kernel,
			$A_{N}(f)$. The resultant signals are then summed to produce the output.

			This method is easier to implement than a wave digital filter as no prior knowledge of the system
			is needed. The amount of computation time needed to process a signal is also considerably less.
			There are however some disadvantages. There is no account for the dependence of the system on the
			amplitude of the input signal. This is shown by \citet{novak2010analysis} who demonstrate how two
			different circuits are emulated with different degrees of accuracy. The is also no definite way of
			deciding what order Hammerstein model to use prior to testing.

\section{Timbre of Nonlinear Distortion}
\label{sec:Excitation-Timbre}
	There is a wealth of research into how low level audio features influence timbre as discussed in Chapter
	\ref{chap:Timbre}. The mappings between low level features and semantic features from the literature can be applied
	to harmonic excitation effects provided the excitation method used can influence the required low level features.
	There may be semantic terms used to describe the timbre of nonlinear distortion which are seldom used to describe
	other timbres. The majority of timbral research does not focus on nonlinear distortion and as such does not
	highlight these semantic terms. There have been some publications specifically discussing the timbral effects of
	nonlinear distortion. Their findings are discussed in this section.

	\citet{marui2005predicting} suggest that one of the primary outcomes of nonlinear distortion is the moving of
	spectral energy between low and high frequencies. They propose that aspects of this effect correlate with the
	descriptors sharpness and brightness. In order to discover other timbral properties of nonlinear distortion they
	performed listening tests in which distorted guitar samples were assessed. The samples were each processed with
	different nonlinear systems and then further processed so that they had matching Zwicker Sharpnesses (an objective
	measure of sharpness \cite{fastl2007psychoacoustics}). This sharpness matching was done so that difference in
	timbre not related to sharpness could be more easily observed. During the listening tests subjects were asked to
	rate the dissimilarity of samples presented in pairs. They were also asked to grade each individual sample on the
	bipolar adjective scales shown in Table \ref{tab:distortionDescriptors}.

	\begin{table}[h!]
		\centering
		\begin{tabular}{|c|C{3cm}cC{3cm}|}
			\hline
			\bf{No.} & \multicolumn{3}{|c|}{\bf{Adjectives}} \tabularnewline
			\hline
			\hline
			1 & dark & $\Longleftrightarrow$ & bright \tabularnewline
			\hline
			2 & rough & $\Longleftrightarrow$ & smooth \tabularnewline
			\hline
			3 & diffuse & $\Longleftrightarrow$ & compact \tabularnewline
			\hline
			4 & thin & $\Longleftrightarrow$ & thick \tabularnewline
			\hline
			5 & sharp & $\Longleftrightarrow$ & dull \tabularnewline
			\hline
			6 & light & $\Longleftrightarrow$ & heavy \tabularnewline
			\hline
			7 & hard & $\Longleftrightarrow$ & soft \tabularnewline
			\hline
			8 & clear & $\Longleftrightarrow$ & muddy \tabularnewline
			\hline
			9 & clamorous & $\Longleftrightarrow$ & calm \tabularnewline
			\hline
			10 & strong & $\Longleftrightarrow$ & weak \tabularnewline
			\hline
			11 & uncomfortably loud & $\Longleftrightarrow$ & comfortable \tabularnewline
			\hline
		\end{tabular}
		\caption{Bipolar adjectives scales used by \citet{marui2005predicting} to assess the perception of
		         distortion.}
		\label{tab:distortionDescriptors}
	\end{table}

	Their results suggest that the differences between different nonlinearities can be described by `thickness' and, to
	a lesser extent, `diffuseness'. These results are also found by a second experiment in which a triadic comparison
	method is used to assess the dissimilarities between samples \citep{marui2005constructing}.

	Both these experiments were conducted in Japanese and the descriptors were translated to English for publication.
	The descriptors were initially chosen during an experiment by \citet{martens2002relating}. From this experiment it
	was shown that speakers of Japanese and speakers of Sinhalese disagree on the how these terms are used to describe
	timbre. It is expected that there will be similar differences in presenting the descriptors to English speakers.

	\citet{tsumoto2015investigating} use factor analysis to identify two factors describing the timbre of distorted
	guitar sounds: `activeness' correlating with attack time and the spectral centroid in the first 30ms of the sound
	and 'brightness' correlating with the spectral centroid after the first 30ms. Various bipolar adjective scales are
	assigned to these factors, `activeness' being described as 'light $\Leftrightarrow$ heavy' and `weak
	$\Leftrightarrow$ strong' and `brightness' being described as `bright $\Leftrightarrow$ dark' and `dull
	$\Leftrightarrow$ sharp'.

	They also use the term `crunch' is used to describe sounds with a moderate amount of distortion. No evidence is
	provided that this terminology is used by a wide range of guitar players, it is merely the language used to
	discriminate certain distortion effects from each other in their paper.

	\citet{wilson2014characterisation} use a different approach to describe the distortion characteristics of popular
	music tracks. A panel of expert listeners rated several samples on two scales: distortion level and distortion
	character (hard or soft). These results were then used to train a classifier for identifying distortion character.
	The features found to best describe the distortion character were various features of the signals probability mass
	function.

	These samples were then used in a listening test in which participants were asked to choose words which describe
	the audio quality of the samples. The most often used word was `distorted', being used to describe the samples
	which were classified has having high distortion levels. This suggests that the concept if distortion is well known
	to listeners. The other descriptors elicited (`punchy', `clear', 'harsh' etc.) do not correlate well with any
	particular group of samples as categorised by the classifier.

	This may be to do the method by which the original sample were classified by the expert panel. The descriptors
	which did not correlate may refer to perceptual dimensions other that distortion level and character. Training the
	classifier to only discriminate samples in these two dimensions removes any information about where the samples lie
	on these additional dimensions.

	From discussion of the previous literature it is apparent that the use of nonlinear systems contribute to a variety
	of aspects of timbre. Distortion effects are widely used as a creative tool in popular music production but the
	language used to describe their application differs across groups of people. The term `distorted' is used to
	describe the timbre of signals which have undergone distortion but does not allow for discrimination between the
	timbres of two distorted signals.

	The other descriptor used in the literature are correlated with features of a systems output signal rather than
	features of the nonlinearity itself. This warrant a need to harmonic excitation systems which give control over
	specific features of the output rather than just the parameters of the underlying nonlinearity. 

\section{Uses of Harmonic Excitation}
\label{sec:Excitation-Uses}
	While harmonic excitation algorithms are best know as a creative effect used by guitarist, they find used is several
	other areas of audio processing. These areas are summarised in Sections \ref{sec:Excitation-Uses-Reconstruction}
	through \ref{sec:Excitation-Uses-Enhancement}. While popular creative distortion algorithms are based on simple
	analogue electronic circuits, algorithms designed for these applications provide more control over the output signal
	in order to achieve their respective tasks.

	\subsection{High Frequency Reconstruction}
	\label{sec:Excitation-Uses-Reconstruction}
		In digital systems it can be beneficial to reduce the amount of data needed to represent an audio signal.
		This is done either to reduce the storage space needed or reduce the data rate required to transmit a
		signal. As a higher data rate is needed to represent high frequencies these frequencies are removed during
		the compression process. A lot of research has been carried out to develop methods by which these high
		frequencies can be estimated during the decoding process. This allows for the bandwidth to be constricted
		for storage or transmission but the full bandwidth to be restored when required.

		When compressing audio data using lossy audio codecs the original signal, along with its high frequency
		components, is available before compression is applied. Parameters describing the high frequency content
		of the signal can be recorded and used to aid in the reconstruction of the high frequencies later
		\citep{dietz2002spectral, friedrich2007spectral}. The high frequency content is estimated from the low
		frequency information and then further shaped using these parameters. This leads to a distinction between
		`blind' and `non blind' methods for reconstructing the high frequencies. A `blind' method uses only the
		information present in the low frequency signal, whereas `non blind' methods make use of recorded
		parameters to increase the accuracy of the reconstruction.

		In the literature several different methods for reconstruction of the higher frequencies are suggested:

		\begin{itemize}
			\item Through the use of a nonlinear device \citep{larsen2002efficient, sha2010high}.
			\item Spectral replication \citep{nagel2010a}, essentially shifting the spectrum into a higher
			      frequency band.
			\item Spectral folding \citep{friedrich2007spectral}, creating a mirror image of the existing
			      spectrum around the highest frequency.
			\item Spectral stretching \citep{nagel2009a}, stretching the existing spectrum to the desired
			      bandwidth.
		\end{itemize}

		These methods could be utilised in creative audio effects. They all introduce new harmonic spectral
		components which may enhance perceptual attributes of a sound. In this situation there is not any
		information about how the higher harmonics might behave so the system will have to operate `blindly'. It is
		worth noting that when using these methods as creative audio effects, the objective is not to try and
		replicate high frequency content which is missing but to generate new high frequency content. The accuracy
		lost through using `blind' bandwidth extension methods is not a great concern for creative timbral
		manipulation work as the generation of new, high order, harmonics does not require the same degree of
		accuracy.

		Information on how these methods can be implemented as simple harmonic exciters is given in Section
		\ref{sec:ExcitationEvaluation-Methods}.

	\subsection{Perceptual Low Frequency Reinforcement}
	\label{sec:Excitation-Uses-Reinforcement}
		Small loudspeakers are incapable of reproducing low frequency signals at sufficient amplitudes. Harmonic
		excitation can be used to psychoacoustically extend the bandwidth of loudspeakers into the lower end of the
		spectrum. Harmonic content is added to the signal in order to evoke the perception of lower pitch sounds.

		The pitch of harmonically structured sounds is the same as that of its fundamental frequency. No energy
		need be present at the fundamental for its pitch to be perceived however. So long as a sufficient
		proportion of the harmonic structure remains the original pitch can still be perceived. This phenomenon is
		commonly referred to as the missing fundamental \citep{plack2005the}.

		This technique is implemented in effects such as Waves' MaxxBass \citep{ben-tzur1999the}. Implementation
		techniques typically involve various filtering stages around a nonlinear stage. Firstly a filter is
		used to isolate the frequencies which need psychoacoustic reinforcement (those below the cutoff
		frequency of the reproduction system). A nonlinearity is then used to produce harmonics of these low
		frequencies which are then possibly shaped by further filtering. 
		
		The largest differences between proposed implementations lie in the nonlinear stage. \citet{gan2001virtual}
		use a psychoacoustically informed amplitude modulation technique to ensure that the harmonics generated
		produce the desired perceived pitch. \citet{larsen2002reproducing} propose the use of simpler nonlinear
		stages, relying on further filtering to shape the spectrum such that the desired pitch is achieved.


	\subsection{Audio Enhancement}
	\label{sec:Excitation-Uses-Enhancement}
		One of the most commonly used enhancement effects is the Aphex Aural Exciter. \citet{shekar2013modeling}
		states that this effect enhances brightness and clarity of a sound through the application of nonlinear
		distortion.

		\citet{chalupper2000aural} ran several tests to determine the effects that the Aural Exciter has on
		different audio samples, concluding that it operates as a `sharpness' maximiser. His analysis of the device
		was very basic, comprising of a frequency response and an analysis of the spectral alterations made to a
		2kHz sine wave. The frequency response shows the liner processing undertaken by the device, showing that it
		mainly amplifies high frequency content. The sine wave response analysis is used to demonstrate the
		nonlinear elements of the device. 

		\citet{dutilleux2011nonlinear} provides a slightly more in depth analysis of the Aural Exciter showing the
		spectral alterations make to a chirp signal. This gives information about how the nonlinear section of the
		device responds to different input frequencies. It is seen that a high level of energy is introduced at the
		second harmonic. It is not mentioned however, what the exciter's parameters were set to during these tests.
		For this reason it is difficult to compare these results with those collected by \citet{chalupper2000aural}
		which seem to disagree as they show a larger amount of new spectral content being introduced. Neither work
		accounts for differences in processing depending on the input amplitude.

