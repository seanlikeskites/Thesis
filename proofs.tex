\documentclass[a4paper]{article}

\usepackage[cm]{fullpage}
\usepackage{amsmath}
\usepackage{parskip}
\renewcommand{\familydefault}{\sfdefault}

\title{Proving Shiz}
\author{}
\date{}

\begin{document}
\maketitle

\section{Rectification Spectrum}
	
	\[ c_{n} = \frac{1}{2\pi} \int_{-\pi}^{\pi} |\sin(x)|e^{-inx} dx \]
	\[ c_{n} = \frac{1}{2\pi} \int_{-\pi}^{\pi} |\sin(x)|\cos(nx) - i|\sin(x)|\sin(nx) dx \]

	$|\sin(x)|\sin(nx)$ is an odd function so:

	\[ \int_{-\pi}^{\pi} i|\sin(x)|\sin(nx) = 0 \]

	Leaving us with:

	\[ c_{n} = \frac{1}{2\pi} \int_{-\pi}^{\pi} |\sin(x)|\cos(nx) dx \]

	$|\sin(x)|\cos(nx)$ is an even function so we can reduce the range of the integral:

	\[ c_{n} = \frac{1}{\pi} \int_{0}^{\pi} \sin(x)\cos(nx) dx \]

	Using the product to sum identity we get:

	\[ c_{n} = \frac{1}{2\pi} \int_{0}^{\pi} \sin((n+1)x) + \sin((1-n)x) dx \]
	\[ c_{n} = \frac{1}{2\pi} \left[ -\frac{\cos((n+1)x)}{n+1} - \frac{\cos((1-n)x)}{1-n} \right]_{0}^{\pi} \]

	When $n$ is even:

	\[ c_{n} = \frac{1}{2\pi} \left( \frac{1}{n+1} + \frac{1}{1-n} + \frac{1}{n+1} + \frac{1}{1-n} \right) \]
	\[ c_{n} = \frac{1}{2\pi} \frac{2(1-n) + 2(n+1)}{1 - n^{2}} \]
	\[ \boxed{c_{n} = \frac{2}{\pi(1 - n^{2})}} \]

	When $n$ is odd:

	\[ c_{n} = \frac{1}{2\pi} \left( -\frac{1}{n+1} - \frac{1}{1-n} + \frac{1}{n+1} + \frac{1}{1-n} \right) \]
	\[ \boxed{c_{n} = 0} \]

\section{Integrator Spectrum}
	
	\[ c_{n} = \frac{kf_{s}}{2\pi} \left( \int_{0}^{\pi} (1-\cos(x))e^{-inx} dx
				              + \int_{\pi}^{2\pi} (3 + \cos(x))e^{-inx} dx 
			                      \right) \]

	Starting with the first integral:
	
	\[ \int_{0}^{\pi} (1-\cos(x))e^{-inx} dx = \int_{0}^{\pi} e^{-inx} - \cos(x)e^{-inx} dx \]
	\[ = \left[ \frac{ie^{-inx}}{n} - \frac{e^{-inx}(\sin(x) - in\cos(x))}{1 - n^{2}} \right]_{0}^{\pi} \]
	\[ = \frac{ie^{-in\pi}}{n} - \frac{ine^{-in\pi}}{1 - n^{2}}  - \frac{i}{n} - \frac{in}{1 - n^{2}} \]
	\[ = \frac{i}{n(1-n^{2})} \left( (1 - n^{2})e^{-in\pi} - n^{2}e^{-in\pi} - (1 - n^{2}) - n^2 \right) \]
	\[ = \frac{i}{n - n^{3}} \left( e^{-in\pi} - 2n^{2}e^{-in\pi} - 1 \right) \]

	Now the second:

	\[ \int_{\pi}^{2\pi} (3+\cos(x))e^{-inx} dx = \int_{\pi}^{2\pi} 3e^{-inx} + \cos(x)e^{-inx} dx \]
	\[ = \left[ \frac{3ie^{-inx}}{n} + \frac{e^{-inx}(\sin(x) - in\cos(x))}{1 - n^{2}} \right]_{\pi}^{2\pi} \]
	\[ = \frac{3i}{n} - \frac{in}{1 - n^{2}} - \frac{3ie^{-in\pi}}{n} - \frac{ine^{-in\pi}}{1 - n^{2}} \]
	\[ = \frac{i}{n(1-n^{2})} \left( 3(1-n^{2}) - n^{2} - 3(1-n^{2})e^{-in\pi} - n^{2}e^{-in\pi} \right) \]
	\[ = \frac{i}{n - n^{3}} \left( 2n^{2}e^{-in\pi} - 3e^{-in\pi} - 4n^{2} + 3 \right) \]

	Summing those we get:

	\[ \frac{i}{n - n^{3}} \left( 2 - 2e^{-in\pi} - 4n^{2} \right) \]

	Thus:

	\[ \boxed{c_{n} = \frac{ikf_{s}}{2\pi} \left( \frac{4n^{2} + 2e^{-in\pi} - 2}{n^{3} - n} \right)} \]

	When $n = 1$:

	\[ c_{1} = \frac{kf_{s}}{2\pi} \left( \int_{0}^{\pi} (1-\cos(x))e^{-ix} dx
				              + \int_{\pi}^{2\pi} (3 + \cos(x))e^{-ix} dx 
			                      \right) \]

	Starting with the first integral:
	
	\[ \int_{0}^{\pi} (1-\cos(x))e^{-ix} dx = \int_{0}^{\pi} e^{-ix} - \cos^{2}(x) + i\cos(x)\sin(x) dx \]
	\[ = \int_{0}^{\pi} e^{-ix} - \frac{1}{2} - \frac{\cos(2x)}{2} + \frac{i\sin(2x)}{2} dx \]
	\[ = \left[ ie^{-ix} - \frac{x}{2} - \frac{\sin(2x)}{4} - \frac{i\cos(2x)}{4} \right]_{0}^{\pi} \]
	\[ = -i - \frac{\pi}{2} - \frac{i}{4} - \left( i - \frac{i}{4} \right) \]
	\[ = -2i - \frac{\pi}{2} \]

	And the second:
	
	\[ \int_{\pi}^{2\pi} (3+\cos(x))e^{-ix} dx = \int_{\pi}^{2\pi} 3e^{-ix} + \cos^{2}(x) - i\cos(x)\sin(x) dx \]
	\[ = \int_{\pi}^{2\pi} 3e^{-ix} + \frac{1}{2} + \frac{\cos(2x)}{2} - \frac{i\sin(2x)}{2} dx \]
	\[ = \left[ 3ie^{-ix} + \frac{x}{2} + \frac{\sin(2x)}{4} + \frac{i\cos(2x)}{4} \right]_{\pi}^{2\pi} \]
	\[ = 3i + \pi + \frac{i}{4} - \left( -3i + \frac{\pi}{2} + \frac{i}{4} \right) \]
	\[ = 6i + \frac{\pi}{2} \]

	Summing gives:

	\[ \boxed{c_{1} = \frac{2ikf_{s}}{\pi}} \]

	When $n = -1$:

	\[ c_{-1} = \frac{kf_{s}}{2\pi} \left( \int_{0}^{\pi} (1-\cos(x))e^{ix} dx
				              + \int_{\pi}^{2\pi} (3 + \cos(x))e^{ix} dx 
			                      \right) \]

	Starting with the first integral:
	
	\[ \int_{0}^{\pi} (1-\cos(x))e^{ix} dx = \int_{0}^{\pi} e^{ix} - \cos^{2}(x) - i\cos(x)\sin(x) dx \]
	\[ = \int_{0}^{\pi} e^{ix} - \frac{1}{2} - \frac{\cos(2x)}{2} - \frac{i\sin(2x)}{2} dx \]
	\[ = \left[ -ie^{ix} - \frac{x}{2} - \frac{\sin(2x)}{4} + \frac{i\cos(2x)}{4} \right]_{0}^{\pi} \]
	\[ = i - \frac{\pi}{2} + \frac{i}{4} - \left( -i + \frac{i}{4} \right) \]
	\[ = 2i - \frac{\pi}{2} \]

	And the second:
	
	\[ \int_{\pi}^{2\pi} (3+\cos(x))e^{ix} dx = \int_{\pi}^{2\pi} 3e^{ix} + \cos^{2}(x) + i\cos(x)\sin(x) dx \]
	\[ = \int_{\pi}^{2\pi} 3e^{ix} + \frac{1}{2} + \frac{\cos(2x)}{2} + \frac{i\sin(2x)}{2} dx \]
	\[ = \left[ -3ie^{ix} + \frac{x}{2} + \frac{\sin(2x)}{4} - \frac{i\cos(2x)}{4} \right]_{\pi}^{2\pi} \]
	\[ = -3i + \pi - \frac{i}{4} - \left( 3i + \frac{\pi}{2} - \frac{i}{4} \right) \]
	\[ = -6i + \frac{\pi}{2} \]

	Summing gives:

	\[ \boxed{c_{-1} = -\frac{2ikf_{s}}{\pi}} \]

	When $n = 0$:

	\[ c_{0} = \frac{kf_{s}}{2\pi} \left( \int_{0}^{\pi} 1-\cos(x) dx
				              + \int_{\pi}^{2\pi} 3 + \cos(x) dx 
			                      \right) \]

	Starting with the first integral:

	\[ \int_{0}^{\pi} 1 - \cos(x) dx = \left[ x + \sin(x) \right]_{0}^{\pi}\]
	\[ = \pi \]

	And the second:

	\[ \int_{\pi}^{2\pi} 3 + \cos(x) dx = \left[ 3x - \sin(x) \right]_{\pi}^{2\pi}\]
	\[ = 6\pi - 3\pi \]
	\[ = 3\pi \]

	Summing gives:

	\[ \boxed{c_{0} = 2kf_{s}} \]

\section{IAP and SSBA Relationship}
	SSBA looks like this:

	\[ y_{SSBA} = \Re \left( z^{n} \right) \]
	\[ y_{SSBA} = \Re \left( (|z|e^{i\arg(z)})^{n} \right) \]
	\[ y_{SSBA} = |z|^{n}\cos(n\arg(z)) \]

	IAP looks like this:

	\[ y_{IAP} = |z|\cos(n\arg(z)) \]

	Thus:

	\[ \boxed{y_{SSBA} = |z|^{n-1}y_{IAP}} \]

\section{Spectral Flatness Manipulation}
	We are concerned with the changes in arithmetic mean. We can simplify by considering the sum of the amplitudes of
	the two scaled groups of spectral components:

	\[ \frac{A_{L}}{m^{2}} + m^{2}A_{H} \]

	For the spectral flatness to increase this should be less than the unscaled amplitudes:

	\[ \frac{A_{L}}{m^{2}} + m^{2}A_{H} < A_{L} + A_{H} \]
	\[ A_{L} + m^{4}A_{H} < (A_{L} + A_{H})m^{2} \]
	\[ A_{H}m^{4} - (A_{L} + A_{H})m^{2} + A_{L} < 0 \]
	\[ (A_{H}m^{2} - A_{L})(m^{2} - 1) < 0 \]
	\[ \frac{A_{L}}{A_{H}} < m^{2} < 1 \]
	\[ \boxed{\sqrt{\frac{A_{L}}{A_{H}}} < m < 1} \]

	The maximum spectral flatness occurs at the minimum arithmetic mean:

	\[ y = \frac{A_{L}}{m^{2}} + m^{2}A_{H} \]
	\[ \frac{dy}{dx} = -2\frac{A_{L}}{m^{3}} + 2A_{H}m \]
	\[ 2A_{H}m^{4} = 2A_{L} \]
	\[ m^{4} = \frac{A_{L}}{A_{H}} \]
	\[ \boxed{m = \sqrt[4]{\frac{A_{L}}{A_{H}}}} \]

\end{document}
